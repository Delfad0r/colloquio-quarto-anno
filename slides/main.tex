\documentclass{beamer}
\usepackage{mystyle}

\begin{document}
\section{Volume simpliciale}
\subsection{Norma e seminorma $\ell^1$}
\begin{frame}{\secname}{\subsecname}
Sia $X$ uno spazio topologico.
\begin{diagram}
\ldots\rar{d_{n+2}}\&C_{n+1}(X)\rar{d_{n+1}}\&C_n(X)\rar{d_n}\&C_{n-1}(X)\rar{d_{n-1}}\&\ldots
\end{diagram}
\begin{block}{Norma $\ell^1$ su $C_n(X)$}
\[
\norm{\sum a_is_i}_1=\sum|a_i|\in(0,+\infty)
\]
\end{block}
\begin{block}{Seminorma $\ell^1$ su $H_n(X)$}
\[
\norm{\alpha}_1=\inf\left\{\norm{c}_1:c\in Z_n(X),[c]=\alpha\right\}\in[0,+\infty)
\]
\end{block}
\end{frame}
\subsection{Classe fondamentale}
\begin{frame}{\secname}{\subsecname}
Sia $M$ una $n$-varietà chiusa orientata.
\begin{itemize}
\item $H_n(M,\ZZ)\iso\ZZ$.
\item L'orientazione fissa un generatore $[M]_\ZZ\in H_n(M,\ZZ)$.
\item Il cambio di coefficienti $C_\bul(M,\ZZ)\to C_\bul(M,\RR)$ induce
\uMap{H_\bul(M,\ZZ)}{H_\bul(M,\RR)}{[M]_\ZZ}{[M]_\RR=[M].}
\end{itemize}
\end{frame}
\subsection{Definizione}
\begin{frame}{\secname}{\subsecname}
\begin{block}{Definizione}
Sia $M$ una $n$-varietà chiusa orientata, $[M]\in H_n(M)$ la sua classe fondamentale. Si chiama \emph{volume simpliciale} il numero reale
\[
\norm{M}=\norm{[M]}_1.
\]
\end{block}
\begin{itemize}
\item Non dipende dall'orientazione $\implies$ è ben definito per varietà chiuse orientabili.
\item Può essere nullo, anche se $[M]\neq 0$.
\end{itemize}
\end{frame}
\subsection{Principio di proporzionalità}
\begin{frame}{\secname}{\subsecname}
\onslide<2>{Lo dimostreremo con un'ipotesi aggiuntiva.}
\begin{block}{Teorema}
Sia $M$ una varietà Riemanniana chiusa\only<2>{ \underline{con curvatura non positiva}}.\\
Allora il rapporto
\[
\frac{\norm{M}}{\Vol(M)}\onslide<2>{=\frac{1}{\norm{[\Vol_{\widetilde M}]^G_c}_\infty}}
\]
dipende solo dal tipo di isometria del rivestimento universale di $M$.
\end{block}
\end{frame}

\section{Applicazioni}
\subsection{Limitazione del grado}
\begin{frame}{\secname}{\subsecname}
\begin{block}{Proposizione}
Siano $M$, $N$ $n$-varietà chiuse orientate, $\map{f}{M}{N}$ una funzione continua di grado $d$. Allora
\[
|d|\cdot\norm{N}\le\norm{M}.
\]
\end{block}
\begin{block}{Corollario}
Sia $\map{f}{M}{M}$ di grado $d\ge 2$. Allora $\norm{M}=0$.
\end{block}
\end{frame}
\subsection{Varietà euclidee}
\begin{frame}{\secname}{\subsecname}
\begin{itemize}
\item L'$n$-toro $(S^1)^n$ ammette endomorfismi di grado arbitrariamente alto; di conseguenza, $\norm{(S^1)^n}=0$.
\item $(S^1)^n$ ammette una metrica piatta, con rivestimento universale isometrico a $\RR^n$.
\item Data una qualunque $n$-varietà chiusa euclidea $M$, vale
\[
\frac{\norm{M}}{\Vol(M)}=\frac{\norm{(S^1)^n}}{\Vol((S^1)^n)}=0
\]
da cui $\norm{M}=0$.
\end{itemize}
\end{frame}
\subsection{Varietà iperboliche}
\begin{frame}{\secname}{\subsecname}
\begin{block}{Teorema}
Sia $M$ una $n$-varietà chiusa iperbolica. Allora
\[
\norm{M}=\frac{\Vol(M)}{v_n},{\color{gray}>0}
\]
dove $v_n$ è il volume dell'$n$-simplesso ideale regolare in $\HH^n$.
\end{block}
\end{frame}
\begin{frame}{\secname}{\subsecname}
\begin{block}{Corollario}
Una varietà chiusa $M$ non può ammettere contemporaneamente una metrica euclidea e una iperbolica.
\end{block}
\begin{block}{Corollario}
Se $M$, $N$ sono varietà iperboliche e $\map{f}{M}{N}$ una funzione continua, allora
\[
|\deg(f)|\le\frac{\norm{M}}{\norm{N}}\tikzmarknode{maps hyperbolic manifolds equal}{{}={}}\frac{\Vol(M)}{\Vol(N)}.
\]
\tikz[overlay,remember picture]\draw[gray!50,-latex] ++(maps hyperbolic manifolds equal) +(-1,-1.2) node[anchor=east,gray] {principio di proporzionalità} to[bend right] +(0,-5pt);
\end{block}
\end{frame}
\subsection{Superfici chiuse}
\begin{frame}{\secname}{\subsecname}
Sia $\Sigma_g$ la superficie chiusa orientabile di genere $g$.
\begin{overlayarea}{\textwidth}{.4\textheight}
\begin{itemize}
\item Per $g\le 1$ vale $\norm{\Sigma_g}=0$. \tikz[baseline=-.7ex]\draw[gray!50,-latex] (1,0) -- (0,0); {\color{gray}$\Sigma_0=S^2$, $\Sigma_1=S^1\times S^1$};
\item Per $g\ge 2$, \alt<1>{$\Sigma_g$ ammette una metrica iperbolica.

\vspace{1cm}
\[
\norm{\Sigma_g}=\frac{\tikzmarknode{closed surfaces num}{\Vol(\Sigma_g)}}{\tikzmarknode{closed surfaces den}{v_2}}=-\frac{2\pi\chi(\Sigma_g)}{\pi}=4g-4
\]
\begin{tikzpicture}[remember picture,overlay]
\draw[gray!50,-latex] ++(closed surfaces num) +(0,9pt) to[bend left] +(1,1) node[gray,anchor=west] {Gauss-Bonnet};
\draw[gray!50,-latex] ++(closed surfaces den) +(0,-6pt) to[bend right] +(1,-1) node[gray,anchor=west,align=left] {area del triangolo\\ideale in $\HH^2$};
\end{tikzpicture}
}{$\norm{\Sigma_g}=4g-4$.}
\item<2> Sia $\map{f}{\Sigma_{g_1}}{\Sigma_{g_2}}$ con $g_1\ge1$, $g_2\ge 2$. Allora
\[
|\deg(f)|\le\frac{g_1-1}{g_2-1}=\frac{\chi(\Sigma_{g_1})}{\chi(\Sigma_{g_2})}.
\]
% See https://math.stackexchange.com/a/2312535/665961
\end{itemize}
\end{overlayarea}
\end{frame}

\section{Principio di proporzionalità}
\subsection{Norma e seminorma $\ell^\infty$}
\begin{frame}{\secname}{\subsecname}
Sia $X$ uno spazio topologico.
\begin{diagram}
\ldots\&\lar["\delta^{n+1}"']C^{n+1}(X)\&\lar["\delta^n"']C^n(X)\&\lar["\delta^{n-1}"']C^{n-1}(X)\&\lar["\delta^{n-2}"']\ldots
\end{diagram}
\begin{block}{Norma $\ell^\infty$ su $C^n(X)$}
\[
\norm{\varphi}_\infty=\sup\left\{|\varphi(c)|:c\in C_n(X),\norm{c}_1\le 1\right\}\in(0,+\infty]
\]
\end{block}
\begin{block}{Seminorma $\ell^\infty$ su $H^n(X)$}
\[
\norm{\beta}_\infty=\inf\left\{\norm{\varphi}_\infty:\varphi\in Z^n(X),[\varphi]=\beta\right\}\in[0,+\infty]
\]
\end{block}
\end{frame}
\subsection{Dualità}
\begin{frame}{\secname}{\subsecname}
Il prodotto di Kronecker è l'applicazione bilineare {\color{gray} ben definita}
\begin{alignat*}{4}
\langle-,-\rangle\colon&H^n(X)&\times H_n(X)&\longrightarrow{}&&\RR\\
&(\;\;\;[\varphi]&,\;\;[z]\;\;\;)&\longmapsto{}&&\varphi(z).
\end{alignat*}

\begin{block}{Proposizione}
Sia $\alpha\in H_n(X)$. Allora
\[
\norm{\alpha}_1=\max\left\{\langle\beta,\alpha\rangle:\beta\in H^n(X),\norm{\beta}_\infty\le 1\right\}.
\]
\end{block}
\end{frame}
\subsection{Coclasse fondamentale}
\begin{frame}{\secname}{\subsecname}
Sia $M$ una $n$-varietà chiusa orientata, $[M]\in H_n(M)$ la sua classe fondamentale.

\begin{itemize}
\item \tikzmarknode{fundamental coclass exists}{Esiste un'unica} classe $[M]^*\in H^n(M)$ tale che
\[
\langle[M]^*,[M]\rangle=1.
\]
\tikz[overlay,remember picture]\draw[gray!50,-latex] ++(fundamental coclass exists) +(-.5,-1) node[anchor=north,gray,align=left] {$H_n(M)\iso\RR$\\$H^n(M)\iso\RR$} to[out=60,in=-100] +(0,-7pt);
\item Per dualità, vale
\[
\norm{M}=\norm{[M]}_1=\frac{1}{\norm{[M]^*}_\infty}.
\]
\end{itemize}
\end{frame}

\subsection{Coomologia $\Gamma$-invariante}
\begin{frame}{\secname}{\subsecname}
\begin{itemize}
\item Sia $M$ una $n$-varietà Riemanniana chiusa e orientata con curvatura non positiva.
\item Sia $\map{p}{\widetilde M}{M}$ il rivestimento universale.
\item Sia $\Gamma=\pi_1(M)$ identificato con $\Aut(\widetilde M,p)$.{\color{gray}$<\Isom^+(\widetilde M)$}
\end{itemize}

$\Gamma$ agisce sul complesso di cocatene $C^\bul(\widetilde M)$.
\begin{block}{Proposizione}
Il rivestimento $p$ induce isomorfismi isometrici
\begin{alignat*}{2}
p^\bul\colon&&C^\bul(M)&\longrightarrow C^\bul(\widetilde M)^\Gamma,\\
H^\bul(p^\bul)\colon&&H^\bul(M)&\longrightarrow H^\bul(C^\bul(\widetilde M)^\Gamma).
\end{alignat*}
\end{block}
\end{frame}
\subsection{Curvatura non negativa}
\begin{frame}{\secname}{\subsecname}
\begin{block}{Teorema (Cartan-Hadamard)}
La mappa esponenziale
\[
\map{\exp_x}{T_x\widetilde M}{\widetilde M}
\]
è un diffeomorfismo
\end{block}
\begin{itemize}
\item $\widetilde M$ è diffeomorfo a $\RR^n$.
\item Per ogni $x,y\in\widetilde M$ esiste un'unica geodetica che li collega.
\item Le parametrizzazioni a velocità costante delle geodetiche dipendono in modo liscio dagli estremi.
\end{itemize}
\end{frame}
\subsection{Mappa di raddrizzamento}
\begin{frame}{\secname}{\subsecname}
Il teorema di Cartan-Hadamard permette di definire il \emph{simplesso dritto} di vertici $x_0,\ldots,x_k\in\widetilde M$.

\begin{minipage}[t]{.5\textwidth}
\begin{itemize}
\item $k=0\leadsto$ $[x_0]$ è lo $0$-simplesso avente immagine $x_0$.
\item $k>0\leadsto$ $[x_0,\ldots,x_k]$ è il ``cono geodetico'' di vertice $x_k$ e base $[x_0,\ldots,x_{k-1}]$.
\end{itemize}
\end{minipage}
\begin{minipage}[t]{.45\textwidth}
\begin{figure}
\begin{tikzpicture}[scale=.35]
\coordinate (x0) at (0,0);
\coordinate (x1) at (10,1);
\coordinate (x2) at (4,5);
\coordinate (x3) at (2,8);
\coordinate (p) at (4,2);
\fill[style between={1}{3}{},style between={3}{4}{dashed},orange!60!black,fill=orange,fill opacity=.4]
(x0)  .. controls ++(20:3) and ++(170:3) ..
(x1) .. controls ++(160:2) and ++(-60:2) ..
(x2) .. controls ++(-100:2) and ++(30:2) .. cycle;
\begin{scope}[decoration={markings,mark=between positions .3 and 1 step .2 with {\arrow{stealth}}}]
\draw (x3) .. controls ++(-90:3) and ++(60:3) .. (x0);
\draw (x3) .. controls ++(-80:4) and ++(165:4) .. (x1);
\draw (x3) .. controls ++(-75:1) and ++(150:1) .. (x2);
\draw[postaction={decorate}] (x3) .. controls ++ (-85:2) and ++(130:2) .. (p);
\end{scope}
\foreach \c in {x0,x1,x2,x3} {\fill (\c) circle[radius=3pt];}
\fill[orange!60!black] (p) circle[radius=3pt];
\foreach \i/\pos in {0/below left,1/right,2/above right,3/above} {\node[\pos] at (x\i) {$x_{\i}$};}
\end{tikzpicture}
\end{figure}
\end{minipage}
\end{frame}
\begin{frame}[fragile]{\secname}{\subsecname}
\begin{center}
\begin{tikzpicture}[scale=.45]
\coordinate (x0) at (0,0);
\coordinate (x1) at (10,1);
\coordinate (x2) at (4,5);
\pgfmathsetseed{9}
\path[random waves=2.5 and 4.5 with name x0] (x0) .. controls ++(20:3) and ++(170:3) .. (x1);
\path[random waves=2.5 and 3.5 with name x1] (x1) .. controls ++(160:2) and ++(-60:2) .. (x2);
\path[random waves=2.5 and 3.5 with name x2] (x2) .. controls ++(-100:2) and ++(30:2) .. (x0);
\newcommand{\drawRoughSimplex}[2]{
\draw[smooth,tension=.5,opacity=#2,orange!60!black,fill=orange,fill opacity=.4*#2] ($(x0)+#1$) \foreach \p/\q in {x0/x1,x1/x2,x2/x0} { -- plot coordinates {($(\p)+#1$) ($(\p-1)+#1$) ($(\p-2)+#1$) ($(\p-3)+#1$) ($(\p-4)+#1$) ($(\p-5)+#1$) ($(\p-6)+#1$) ($(\p-7)+#1$) ($(\p-8)+#1$) ($(\p-9)+#1$) ($(\q)+#1$)} };}
\newcommand{\drawSmoothSimplex}[2]{
\draw[black,opacity=#2,blue!30!black,fill=blue,fill opacity=.4*#2]
($(x0)+#1$) .. controls ++(20:3) and ++(170:3) ..
($(x1)+#1$) .. controls ++(160:2) and ++(-60:2) ..
($(x2)+#1$) .. controls ++(-100:2) and ++(30:2) .. cycle;
}
\tikzmath{\h1=7;}
\onslide<2->{
\foreach \p in {x0,x1,x2} {
\draw[thin,gray,dashed] ++(\p) -- +(0,\h1);
}
}
\onslide<1>{\drawRoughSimplex{(0,0)}{1}}
\onslide<2->{\drawRoughSimplex{(0,0)}{.5}}
\onslide<2>{\drawRoughSimplex{(0,\h1)}{1}}
\onslide<3->{\drawRoughSimplex{(0,\h1)}{.5}}
\onslide<3>{\drawSmoothSimplex{(0,\h1)}{1}}
\onslide<4->{\drawSmoothSimplex{(0,\h1)}{.5}}
\onslide<4>{\drawSmoothSimplex{(0,0)}{1}}

\foreach \p in {x0,x1,x2} {
\fill[alt=<2-3>{orange!60!gray}{alt=<1>{orange!60!black}{blue!30!black}}] (\p) circle[radius=2pt];
\onslide<2->{\fill[alt=<2>{orange!60!black}{alt=<3>{blue!30!black}{blue!30!gray}}] ($(\p)+(0,\h1)$) circle[radius=2pt];}
}
\node (M) at ($(x1)+(3,2)$) {$M$};
\node (Mtilde) at ($(x1)+(3,2+\h1)$) {$\widetilde M$};
\draw[->,shorten <=10pt,shorten >=10pt] (Mtilde.south) -- (M.north) node[midway,right] {$p$};
\onslide<1-3>{\node[alt=<1>{opacity=1}{opacity=.5},orange!60!black] (s) at ($1/3*(x0)+1/3*(x1)+1/3*(x2)+(0,.5)$) {$s$};}
\onslide<2>{\node[alt=<2>{opacity=1}{opacity=.5},orange!60!black] (stilde) at ($(s)+(0,\h1)$) {$\widetilde s$};}
\onslide<3-4>{\node[alt=<3>{opacity=1}{opacity=.5},blue!30!black] (strtilde) at ($(Mtilde)+(-17,0)$) {$[\widetilde s(e_0),\ldots,\widetilde s(e_k)]$};}
\onslide<4>{\node[blue!30!black] at ($(strtilde)+(0,-\h1)$) {$\str_k(s)$};}
\end{tikzpicture}
\end{center}
\end{frame}
\subsection*{Cociclo volume}
\begin{frame}{\secname}{\subsecname}
Per ogni $n$-simplesso $\map{s}{\Delta^n}{M}$ definiamo
\[
\Vol_M(s)=\int_{\str_n(s)}\omega_M.
\]
\begin{itemize}
\item $\Vol_M(d_{n+1}(s))\tikzmarknode{volume cocycle stokes}{{}={}}0$, dunque è un cociclo.
\tikz[overlay,remember picture]\draw[gray!50,-latex] ++(volume cocycle stokes) +(-1,.8) node[gray,anchor=east] {Stokes} to[bend left] +(0,5pt);
\item Definisce una classe $[\Vol_M]\in H^n(M)$ in coomologia.
\item Vale \tikzmarknode{volume cocycle VolM}{$[\Vol_M]=\Vol(M)\cdot[M]^*$.}
\end{itemize}
\vspace{.5cm}
\[
\color{gray} \norm{M}=\frac{1}{\norm{[M]^*}_\infty}\tikzmarknode{volume cocycle equal}{{}={}}\frac{\Vol(M)}{\norm{[\Vol_M]}_\infty}
\]
\tikz[remember picture,overlay]\draw[gray!50,-latex] (volume cocycle VolM.east) to[out=-30,in=90] ($(volume cocycle equal)+(0,5pt)$);
\end{frame}
\begin{frame}{\secname}{\subsecname}
\begin{diagram}[row sep=tiny,remember picture]
C^\bul(M)
\rar["p^\bul","\iso"']\&
C^\bul(\widetilde M)^\Gamma\&
|[alias=volume cocycle tl]|{\visible<9->{\lar[visible on=<9->,hook']C^\bul(\widetilde M)^G}}\\
{\visible<3->{\Vol_M}}
\rar[visible on=<5->,mapsto]
\&
|[alias=volume cocycle VolMtilde]|{\visible<5->{\Vol_{\widetilde M}}}
\&
{\visible<9->{\lar[visible on=<9->,equal]\Vol_{\widetilde M}}}
\\\phantom{.}\\\phantom{.}\\
{\visible<2->{H^\bul(M)}}
\rar[visible on=<2->,"\iso"]
\&
{\visible<2->{H^\bul(C^\bul(\widetilde M)^\Gamma)}}
{\visible<13>{\rar[visible on=<13>,gray,"?",bend left]}}
\&
{\visible<10->{\lar[visible on=<10->,"\alt<12>{?}{}"']H^\bul(C^\bul(\widetilde M)^G)}}
\\
{\visible<4->{\left[\Vol_M\right]}}
\rar[visible on=<7->,mapsto]
\&
{\visible<7->{\left[\Vol_{\widetilde M}\right]^\Gamma}}
\&
|[alias=volume cocycle br]|{\visible<10->{\lar[visible on=<10->,mapsto]\left[\Vol_{\widetilde M}\right]^G}}
\end{diagram}
\begin{onslide}<6-10>
\[
\color{gray}\tikzmark{volume cocycle integral}{\visible<6->{\Vol_{\widetilde M}(s)=\int_{\str_n(s)}\omega_{\widetilde M}}}{\visible<8->{\implies\text{è $G$-invariante}}}
\]
\tikz[overlay,remember picture]\draw[-latex,gray!50] ($(volume cocycle VolMtilde.center)+(-10pt,-7pt)$) .. controls ++(-5,-.5) and ++(-3,0) .. ([shift={(-4pt,4pt)}]pic cs:volume cocycle integral);
\end{onslide}
\begin{onslide}<11>
\begin{tikzpicture}[overlay,remember picture]
\draw[gray!50,rounded corners=10pt] ($(volume cocycle tl.north west)+(-.3cm,.2cm)$) rectangle ($(volume cocycle br.south east)+(.3cm,-.2cm)$);
\draw[gray!50,-latex] ++($(volume cocycle br.south)+(0,-.2cm)$) to[out=-120,in=0] +(-2,-.7) node[gray,anchor=east] {non dipende da $M$};
\end{tikzpicture}
\end{onslide}
\end{frame}
\subsection*{Coomologia continua}
\begin{frame}{\secname}{\subsecname}
Sia $S_k(M)=\{\Delta^k\to M\}$ lo spazio dei $k$-simplessi singolari, munito della topologia compatta-aperta.
\begin{block}{Definizione}
Una cocatena $\varphi\in C^k(M)$ è \emph{continua} se la restrizione
\[
\map{\varphi}{S_k(M)}{\RR}
\]
è continua.
\end{block}
\begin{itemize}
\item Si definisce $C^k_c(M)=\{\varphi\in C^k(M):\text{$\varphi$ è continua}\}$.
\item $C^\bul_c(M)$ è un sottocomplesso di $C^\bul(M)$.
\item Si pone $H^\bul_c(M)=H^\bul(C^\bul_c(M))$.
\end{itemize}
\end{frame}
\begin{frame}{\secname}{\subsecname}
\begin{block}{Proposizione}
L'inclusione $C^\bul_c(M)\hookrightarrow C^\bul(M)$ induce un isomorfismo isometrico
\[
H^\bul_c(M)\iso H^\bul(M).
\]
\end{block}
\begin{block}{Proposizione}
L'inclusione $C^\bul_c(\widetilde M)^G\hookrightarrow C^\bul_c(\widetilde M)^\Gamma$ induce un'immersione isometrica
\[
H^\bul(C^\bul_c(\widetilde M)^G)\longhookrightarrow H^\bul(C^\bul_c(\widetilde M)^\Gamma).
\]
\end{block}
\end{frame}
\subsection{Isomorfismi isometrici}
\begin{frame}{\secname}{\subsecname}
\begin{diagram}[row sep=tiny]
\only<1-5>{
{\visible<1-4>{C^\bul(M)}}\&
{\visible<2-4>{\lar[visible on=<2-4>,hook']C^\bul_c(M)\rar[visible on=<3-4>,"p^\bul"]}}\&
{\visible<3-4>{C^\bul_c(\widetilde M)^\Gamma}}\&
{\visible<4>{\lar[visible on=<4>,hook']C^\bul_c(\widetilde M)^G}}\\
{\visible<1-4>{\Vol_M\rar[visible on=<2-4>,equal]}}\&
{\visible<2-4>{\Vol_M\rar[visible on=<3-4>,mapsto]}}\&
{\visible<3-4>{\Vol_{\widetilde M}\rar[visible on=<4>,equal]}}\&
{\visible<4>{\Vol_{\widetilde M}}}\\
\phantom{.}\\\phantom{.}\\}
H^\bul(M)\&
{\visible<2->{\lar[visible on=<2->,"\iso"']\tikzmarknode{isometric isomorphisms iso}{H^\bul_c(M)}\rar[visible on=<3->,"\iso"]}}\&
{\visible<3->{H^\bul(C^\bul_c(\widetilde M)^\Gamma)}}\&
{\visible<4->{\lar[visible on=<4->,hook']H^\bul(C^\bul_c(\widetilde M)^G)}}\\
\left[\Vol_M\right]\&
{\visible<2->{\lar[visible on=<2->,mapsto]\left[\Vol_M\right]_c\rar[visible on=<3->,mapsto]}}\&
{\visible<3->{\left[\Vol_{\widetilde M}\right]^\Gamma_c}}\&
{\visible<4->{\lar[visible on=<4->,mapsto]\left[\Vol_{\widetilde M}\right]^G_c}}\\\phantom{.}
\end{diagram}
\tikz[overlay,remember picture,visible on=<3>]\draw[gray!50,-latex] ++($(isometric isomorphisms iso)+(1.25,0)$) +(-1,1) node[gray,anchor=east] {anche in $H^\bul_c$} to[bend left] +(0,10pt);
\begin{onslide}<7->
Possiamo infine calcolare:
\[
\onslide<8->{\norm{M}\tikzmarknode{isometric isomorphisms duality}{{}={}}\frac{1}{\norm{[M]^*}_\infty}\tikzmarknode{isometric isomorphisms vol}{{}={}}\frac{\Vol(M)}{\norm{[\Vol_M]}_\infty}}\onslide<9->{=\frac{\Vol(M)}{\norm{[\Vol_{\widetilde M}]^G_c}_\infty}.}
\]
\begin{tikzpicture}[overlay,remember picture,visible on=<8>]
\draw[gray!50,-latex] ++(isometric isomorphisms duality) +(-1,-1) node[gray,anchor=east] {dualità} to [bend right] +(0,-5pt);
\draw[gray!50,-latex] ++(isometric isomorphisms vol) +(1,-1) node[gray,anchor=west] {$[Vol_M]=\Vol(M)\cdot[M]^*$} to [bend left] +(0,-5pt);
\end{tikzpicture}
\end{onslide}
\end{frame}
\subsection{Varietà iperboliche}
\begin{frame}{\secname}{\subsecname}
\only<1>{
Sia $M$ una $n$-varietà iperbolica chiusa.
\begin{itemize}
\item Il rivestimento universale è $\widetilde M=\HH^n$.
\item Obiettivo: stimare $\tikzmarknode{hyperbolic manifolds vol}{\norm{[\Vol_{\HH^n}]^G_c}_\infty}$.
\tikz[overlay,remember picture]\draw[gray!50,-latex] ++(hyperbolic manifolds vol.east) +(-4pt,2pt) to[out=-30,in=90] +(.3,-1) node[inner sep=-3pt,gray,anchor=north west] {$\begin{aligned}\Vol_{\HH^n}(s)&=\int_{\str_n(s)}\omega_{\HH^n}\le v_n\end{aligned}$};
\item I simplessi dritti sono \emph{geodetici}.
\end{itemize}
\vspace{.5cm}
}
\begin{block}{Teorema}
Sia $\Delta$ un $n$-simplesso geodetico in $\HH^n$. Allora
\[
\Vol(\Delta)\le v_n,
\]
dove $v_n$ è il volume del $n$-simplesso regolare ideale.
\end{block}
\only<2>{
\begin{itemize}
\item Per ogni $x\in C_n(\HH^n)$ vale
\[
|\Vol_{\HH^n}(c)|\le v_n\cdot\norm{c}_1\implies \norm{\Vol_{\HH^n}}_\infty\le v_n.
\]
\item Allora
\[
\tikzmarknode{hyperbolic manifolds vn}{\textstyle\norm{[\Vol_{\HH^n}]^G_c}_\infty}\le\norm{\Vol_{\HH^n}}_\infty\le v_n.
\]
\tikz[overlay,remember picture]\draw[gray!50,-latex] ++(hyperbolic manifolds vn.south west) +(-2pt,7pt) to[bend right] +(-1.8,0) node[gray,anchor=north] {$\displaystyle\norm{M}\ge\frac{\Vol(M)}{v_n}$};
\end{itemize}
}
\end{frame}
\end{document}