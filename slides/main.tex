\documentclass{beamer}
\usepackage{mystyle}

\begin{document}
\section{Volume simpliciale}
\subsection{Norma e seminorma $\ell^1$}
\begin{frame}{\secname}{\subsecname}
Sia $X$ uno spazio topologico.
\begin{diagram}
\ldots\rar{d_{n+2}}\&C_{n+1}(X)\rar{d_{n+1}}\&C_n(X)\rar{d_n}\&C_{n-1}(X)\rar{d_{n-1}}\&\ldots
\end{diagram}
\begin{block}{Norma $\ell^1$ su $C_n(X)$}
\[
\norm{\sum a_is_i}_1=\sum|a_i|\in(0,+\infty)
\]
\end{block}
\begin{block}{Seminorma $\ell^1$ su $H_n(X)$}
\[
\norm{\alpha}_1=\inf\left\{\norm{c}_1:c\in Z_n(X),[c]=\alpha\right\}\in[0,+\infty)
\]
\end{block}
\end{frame}
\subsection{Classe fondamentale}
\begin{frame}{\secname}{\subsecname}
Sia $M$ una $n$-varietà chiusa orientata.
\begin{itemize}
\item $H_n(M,\ZZ)\iso\ZZ$.
\item L'orientazione fissa un generatore $[M]_\ZZ\in H_n(M,\ZZ)$.
\item Il cambio di coefficienti $C_\bul(M,\ZZ)\to C_\bul(M,\RR)$ induce
\uMap{H_\bul(M,\ZZ)}{H_\bul(M,\RR)}{[M]_\ZZ}{[M]_\RR=[M].}
\end{itemize}
\end{frame}
\subsection{Definizione}
\begin{frame}{\secname}{\subsecname}
\begin{block}{Definizione}
Sia $M$ una $n$-varietà chiusa orientata, $[M]\in H_n(M)$ la sua classe fondamentale. Si chiama \emph{volume simpliciale} il numero reale
\[
\norm{M}=\norm{[M]}_1.
\]
\end{block}
\begin{itemize}
\item Non dipende dall'orientazione $\implies$ è ben definito per varietà chiuse orientabili.
\item Può essere nullo, anche se $[M]\neq 0$.
\end{itemize}
\end{frame}
\subsection{Principio di proporzionalità}
\begin{frame}{\secname}{\subsecname}
\onslide<2>{Lo dimostreremo con un'ipotesi aggiuntiva.}
\begin{block}{Teorema}
Sia $M$ una varietà Riemanniana chiusa\only<2>{ con curvatura non positiva}.\\
Allora il rapporto
\[
\frac{\norm{M}}{\Vol(M)}\onslide<2>{=\frac{1}{\norm{[\Vol_{\widetilde M}]^G_c}_\infty}}
\]
dipende solo dal tipo di isometria del rivestimento universale di $M$.
\end{block}
\end{frame}

\section{Applicazioni}
\subsection{Limitazione del grado}
\begin{frame}{\secname}{\subsecname}
\begin{block}{Proposizione}
Siano $M$, $N$ $n$-varietà chiuse orientate, $\map{f}{M}{N}$ una funzione continua di grado $d$. Allora
\[
|d|\cdot\norm{N}\le\norm{M}.
\]
\end{block}
\begin{block}{Corollario}
Sia $\map{f}{M}{M}$ di grado $d\ge 2$. Allora $\norm{M}=0$.
\end{block}
\end{frame}
\subsection{Varietà euclidee}
\begin{frame}{\secname}{\subsecname}
\begin{itemize}
\item L'$n$-toro $(S^1)^n$ ammette endomorfismi di grado arbitrariamente alto; di conseguenza, $\norm{(S^1)^n}=0$.
\item $(S^1)^n$ ammette una metrica piatta, con rivestimento universale isometrico a $\RR^n$.
\item Data una qualunque $n$-varietà chiusa euclidea $M$, vale
\[
\frac{\norm{M}}{\Vol(M)}=\frac{\norm{(S^1)^n}}{\Vol((S^1)^n)}=0
\]
da cui $\norm{M}=0$.
\end{itemize}
\end{frame}
\subsection{Varietà iperboliche}
\begin{frame}{\secname}{\subsecname}
\begin{block}{Teorema}
Sia $M$ una $n$-varietà chiusa iperbolica. Allora
\[
\norm{M}=\frac{\Vol(M)}{v_n},
\]
dove $v_n$ è il volume dell'$n$-simplesso ideale regolare in $\HH^n$.
\end{block}
\begin{block}{Corollario}
Una varietà chiusa $M$ non può ammettere contemporaneamente una metrica euclidea e una iperbolica.
\end{block}
\end{frame}
\subsection{Mappe fra superfici}
\begin{frame}{\secname}{\subsecname}
Sia $\Sigma_g$ la superficie chiusa orientabile di genere $g$.
\begin{itemize}
\item Per $g\le 1$ vale $\norm{\Sigma_g}=0$
\item Per $g\ge 2$, $\Sigma_g$ ammette una metrica iperbolica.
\[
\norm{\Sigma_g}=\frac{\Vol(\Sigma_g)}{v_2}=-\frac{2\pi\chi(\Sigma_g)}{\pi}=4g-4.
\]
\item Sia $\map{f}{\Sigma_{g_1}}{\Sigma_{g_2}}$, con $g_1\ge 1$, $g_2\ge 2$. Allora
\[
|\deg(f)|\le\frac{\norm{\Sigma_{g_1}}}{\norm{\Sigma_{g_2}}}=\frac{g_1-1}{g_2-1}.
\]
% See https://math.stackexchange.com/a/2312535/665961
\end{itemize}
\end{frame}

\section{Principio di proporzionalità}
\subsection{Norma e seminorma $\ell^\infty$}
\begin{frame}{\secname}{\subsecname}
Sia $X$ uno spazio topologico
\begin{diagram}
\ldots\&\lar["\delta^{n+1}"']C^{n+1}(X)\&\lar["\delta^n"']C^n(X)\&\lar["\delta^{n-1}"']C^{n-1}(X)\&\lar["\delta^{n-2}"']\ldots
\end{diagram}
\begin{block}{Norma $\ell^\infty$ su $C^n(X)$}
\[
\norm{\varphi}_\infty=\sup\left\{|\varphi(c)|:c\in C_n(X),\norm{c}_1\le 1\right\}\in(0,+\infty]
\]
\end{block}
\begin{block}{Seminorma $\ell^\infty$ su $H^n(X)$}
\[
\norm{\beta}_\infty=\inf\left\{\norm{\varphi}_\infty:\varphi\in Z^n(X),[\varphi]=\beta\right\}\in[0,+\infty]
\]
\end{block}
\end{frame}
\subsection{Dualità}
\begin{frame}{\secname}{\subsecname}
Il prodotto di Kronecker è l'applicazione bilineare
\Map{\langle-,-\rangle}{H^n(X)\times H_n(X)}{\RR}{([\varphi],[z])}{\varphi(z).}

\begin{block}{Proposizione}
Sia $\alpha\in H_n(X)$. Allora
\[
\norm{\alpha}_1=\max\left\{\langle\beta,\alpha\rangle:\beta\in H^n(X),\norm{\beta}_\infty\le 1\right\}.
\]
\end{block}
\end{frame}
\subsection{Coclasse fondamentale}
\begin{frame}{\secname}{\subsecname}
Sia $M$ una $n$-varietà chiusa orientata, $[M]\in H_n(M)$ la sua classe fondamentale.

\begin{itemize}
\item Esiste un'unica classe $[M]^*\in H^n(M)$ tale che
\[
\langle[M]^*,[M]\rangle=1.
\]
\item Per dualità, vale
\[
\norm{M}=\norm{[M]}_1=\frac{1}{\norm{[M]^*}_\infty}.
\]
\end{itemize}
\end{frame}

\subsection{Coomologia $\Gamma$-invariante}
\begin{frame}{\secname}{\subsecname}
\begin{itemize}
\item Sia $M$ una $n$-varietà Riemanniana chiusa e orientata con curvatura non positiva.
\item Sia $\map{p}{\widetilde M}{M}$ il rivestimento universale.
\item Sia $\Gamma=\pi_1(M)$ identificato con $\Aut(\widetilde M,p)$.
\end{itemize}

$\Gamma$ agisce sul complesso di cocatene $C^\bul(\widetilde M)$.
\begin{block}{Proposizione}
Il rivestimento $p$ induce isomorfismi isometrici
\begin{alignat*}{2}
p^\bul\colon&&C^\bul(M)&\longrightarrow C^\bul(\widetilde M)^\Gamma,\\
H^\bul(p^\bul)\colon&&H^\bul(M)&\longrightarrow H^\bul(C^\bul(\widetilde M)^\Gamma).
\end{alignat*}
\end{block}
\end{frame}
\subsection{Curvatura non negativa}
\begin{frame}{\secname}{\subsecname}
\begin{block}{Teorema (Cartan-Hadamard)}
La mappa esponenziale
\[
\map{\exp_p}{T_p\widetilde M}{\widetilde M}
\]
è un diffeomorfismo
\end{block}
\begin{itemize}
\item $\widetilde M$ è diffeomorfo a $\RR^n$.
\item Per ogni $x,y\in\widetilde M$ esiste un'unica geodetica che li collega.
\item Le parametrizzazioni a velocità costante delle geodetiche dipendono in modo liscio dagli estremi.
\end{itemize}
\end{frame}
\subsection{Mappa di raddrizzamento}
\begin{frame}{\secname}{\subsecname}
Il teorema di Cartan-Hadamard permette di definire il \emph{simplesso dritto} di vertici $x_0,\ldots,x_k\in\widetilde M$.

\begin{minipage}[t]{.6\textwidth}
\begin{itemize}
\item $k=0\leadsto$ $[x_0]$ è lo $0$-simplesso avente immagine $x_0$.
\item $k>0\leadsto$ $[x_0,\ldots,x_k]$ è il ``cono geodetico'' di vertice $x_k$ e base $[x_0,\ldots,x_{k-1}]$.
\end{itemize}
\end{minipage}
\begin{minipage}[t]{.35\textwidth}
\begin{itemize}
\item \huge Inserire figura
\end{itemize}
\end{minipage}
\end{frame}
\begin{frame}{\secname}{\subsecname}
% Definire direttamente str senza tilde
Per ogni $k$-simplesso singolare $\map{s}{\Delta^k}{\widetilde M}$ definiamo il simplesso
\[
\tstr_k(s)=[s(e_0),\ldots,s(e_k)].
\]

\begin{minipage}[t]{.6\textwidth}
\begin{itemize}
\item $\map{\tstr_\bul}{C_\bul(\widetilde M)}{C_\bul(\widetilde M)}$ è un morfismo di complessi $\Gamma$-equivariante.
\item Induce $\map{\str_\bul}{C_\bul(M)}{C_\bul(M)}$.
\item Entrambi i morfismi sono omotopi all'identità.
\end{itemize}
\end{minipage}
\begin{minipage}[t]{.35\textwidth}
\begin{itemize}
\item\huge RIFARE
\end{itemize}
\end{minipage}
\end{frame}
\subsection*{Cociclo volume}
\begin{frame}{\secname}{\subsecname}
Per ogni $n$-simplesso $\map{s}{\Delta^n}{M}$ definiamo
\[
\Vol_M(s)=\int_{\str_n(s)}\omega_M,
\]
dove $\omega_M$ è la forma volume.
\begin{itemize}
\item $\Vol_M(d_{n+1}(s))=0$, dunque è un cociclo.
\item Definisce una classe $[\Vol_M]\in H^n(M)$ in coomologia.
\item Vale 
\[
[\Vol_M]=\Vol(M)\cdot[M]^*.
\]
\end{itemize}
\end{frame}
\begin{frame}{\secname}{\subsecname}
\begin{diagram}[row sep=tiny,remember picture]
C^\bul(M)
\rar["p^\bul","\iso"']\&
C^\bul(\widetilde M)^\Gamma\&
{\visible<9->{\lar[visible on=<9->,hook']C^\bul(\widetilde M)^G}}\\
{\visible<3->{\Vol_M}}
\rar[visible on=<5->,mapsto]
\&
|[alias=VolMtilde]|{\visible<5->{\Vol_{\widetilde M}}}
\&
{\visible<9->{\lar[visible on=<9->,equal]\Vol_{\widetilde M}}}
\\\phantom{.}\\\phantom{.}\\
{\visible<2->{H^\bul(M)}}
\rar[visible on=<2->,"\iso"]
\&
{\visible<2->{H^\bul(C^\bul(\widetilde M)^\Gamma)}}
\&
{\visible<10->{\lar[visible on=<10->,"\alt<11->{?}{}"']H^\bul(C^\bul(\widetilde M)^G)}}
\\
{\visible<4->{\left[\Vol_M\right]}}
\rar[visible on=<7->,mapsto]
\&
{\visible<7->{\left[\Vol_{\widetilde M}\right]^\Gamma}}
\&
{\visible<10->{\lar[visible on=<10->,mapsto]\left[\Vol_{\widetilde M}\right]^G}}
\end{diagram}
\begin{center}
\begin{tikzpicture}[remember picture]
\node[gray] (testtest) {$\displaystyle{\visible<6->{\Vol_{\widetilde M}(s)=\int_{\str_n(s)}\omega_{\widetilde M}}}{\visible<8->{\implies\text{è $G$-invariante}}}$};
\end{tikzpicture}
\end{center}
\begin{onslide}<6->
\begin{tikzpicture}[overlay,remember picture]
\draw[-latex,gray!50] (VolMtilde) to[out=-170,in=180,out looseness=2,in looseness=1.5] (testtest.west);
\end{tikzpicture}
\end{onslide}
\end{frame}
\subsection*{Coomologia continua}
\begin{frame}{\secname}{\subsecname}
Sia $S_k(M)=\{\Delta^k\to M\}$ lo spazio dei $k$-simplessi singolari, munito della topologia compatta-aperta.
\begin{block}{Definizione}
Una cocatena $\varphi\in C^k(M)$ è \emph{continua} se la restrizione
\[
\map{\varphi}{S_k(M)}{\RR}
\]
è continua.
\end{block}
\begin{itemize}
\item $C^\bul_c(M)$ è un sottocomplesso di $C^\bul(M)$.
\item Si pone $H^\bul_c(M)=H^\bul(C^\bul_c(M))$.
\end{itemize}
\end{frame}
\begin{frame}{\secname}{\subsecname}
\begin{block}{Proposizione}
L'inclusione $C^\bul_c(M)\hookrightarrow C^\bul(M)$ induce un isomorfismo isometrico
\[
H^\bul_c(M)\iso H^\bul(M).
\]
\end{block}
\begin{block}{Proposizione}
L'inclusione $C^\bul_c(\widetilde M)^G\hookrightarrow C^\bul_c(\widetilde M)^\Gamma$ induce un'immersione isometrica
\[
H^\bul(C^\bul_c(\widetilde M)^G)\longhookrightarrow H^\bul(C^\bul_c(\widetilde M)^\Gamma).
\]
\end{block}
\end{frame}
\subsection{Isomorfismi isometrici}
\begin{frame}{\secname}{\subsecname}
\begin{diagram}[row sep=tiny]
\only<1-5>{
{\visible<1-4>{C^\bul(M)}}\&
{\visible<2-4>{\lar[visible on=<2-4>,hook']C^\bul_c(M)\rar[visible on=<3-4>,"p^\bul"]}}\&
{\visible<3-4>{C^\bul_c(\widetilde M)^\Gamma}}\&
{\visible<4>{\lar[visible on=<4>,hook']C^\bul_c(\widetilde M)^G}}\\
{\visible<1-4>{\Vol_M\rar[visible on=<2-4>,equal]}}\&
{\visible<2-4>{\Vol_M\rar[visible on=<3-4>,mapsto]}}\&
{\visible<3-4>{\Vol_{\widetilde M}\rar[visible on=<4>,equal]}}\&
{\visible<4>{\Vol_{\widetilde M}}}\\
\phantom{.}\\\phantom{.}\\}
H^\bul(M)\&
{\visible<2->{\lar[visible on=<2->,"\iso"']H^\bul_c(M)\rar[visible on=<3->,"\iso"]}}\&
{\visible<3->{H^\bul(C^\bul_c(\widetilde M)^\Gamma)}}\&
{\visible<4->{\lar[visible on=<4->,hook']H^\bul(C^\bul_c(\widetilde M)^G)}}\\
\left[\Vol_M\right]\&
{\visible<2->{\lar[visible on=<2->,mapsto]\left[\Vol_M\right]_c\rar[visible on=<3->,mapsto]}}\&
{\visible<3->{\left[\Vol_{\widetilde M}\right]^\Gamma_c}}\&
{\visible<4->{\lar[visible on=<4->,mapsto]\left[\Vol_{\widetilde M}\right]^G_c}}\\\phantom{.}
\end{diagram}
\begin{onslide}<7->
Possiamo infine calcolare:
\vspace{-.5cm}
\begin{center}
\[
\onslide<8->{\norm{M}=\frac{1}{\norm{[M]^*}_\infty}}\onslide<9->{\underset{\tikzmark{equal}}{=}\frac{\Vol(M)}{\norm{[\Vol_M]}_\infty}}\onslide<10->{=\frac{\Vol(M)}{\norm{[\Vol_{\widetilde M}]^G_c}_\infty}.}
\]
\begin{tikzpicture}[remember picture,overlay,visible on=<9>]
\node[gray] (expl) at ([shift={(0,-1cm)}]pic cs:equal) {$[\Vol_M]=\Vol(M)\cdot[M]^*$};
\draw[-latex,gray!50] (expl) to (pic cs:equal);
\end{tikzpicture}
%\begin{tikzpicture}
%\node {$\displaystyle\onslide<8->{\norm{M}=\frac{1}{\norm{[M]^*}_\infty}}\onslide<9->=\frac{\Vol(M)}{\norm{[\Vol_M]}_\infty}}\onslide<10->{=\frac{\Vol(M)}{\norm{[\Vol_{\widetilde M}]^G_c}_\infty}.}$};
%\end{tikzpicture}
\end{center}
\end{onslide}
\end{frame}
\end{document}