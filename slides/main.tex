\documentclass{beamer}
\usepackage{mystyle}

\begin{document}
\section{Volume simpliciale}
\subsection{Norma e seminorma $\ell^1$}
\begin{frame}{\secname}{\subsecname}
Sia $X$ uno spazio topologico.
\begin{diagram}
\ldots\rar{d_{n+2}}\&C_{n+1}(X)\rar{d_{n+1}}\&C_n(X)\rar{d_n}\&C_{n-1}(X)\rar{d_{n-1}}\&\ldots
\end{diagram}
\begin{block}{Norma $\ell^1$ su $C_n(X)$}
\[
\norm{\sum a_is_i}_1=\sum|a_i|
\]
\end{block}
\begin{block}{Seminorma $\ell^1$ su $H_n(X)$}
\[
\norm{c}_1=\inf\left\{\norm{z}_1:z\in Z_n(X),[z]=c\right\}
\]
\end{block}
\end{frame}
\subsection{Classe fondamentale}
\begin{frame}{\secname}{\subsecname}
Sia $M$ una $n$-varietà chiusa orientata.
\begin{itemize}
\item $H_n(M,\ZZ)\iso\ZZ$.
\item L'orientazione fissa un generatore $[M]_\ZZ\in H_n(M,\ZZ)$.
\item Il cambio di coefficienti $C_\bul(M,\ZZ)\to C_\bul(M,\RR)$ induce
\uMap{H_\bul(M\ZZ)}{H_\bul(M,\RR)}{[M]_\ZZ}{[M]_\RR=[M].}
\end{itemize}
\end{frame}
\subsection{Definizione}
\begin{frame}{\secname}{\subsecname}
\begin{block}{Definizione}
Sia $M$ una $n$-varietà chiusa orientata, $[M]\in H_n(M)$ la sua classe fondamentale. Si chiama \emph{volume simpliciale} il numero reale
\[
\norm{M}=\norm{[M]}_1.
\]
\end{block}
\begin{itemize}
\item Non dipende dall'orientazione $\implies$ è ben definito per varietà chiuse orientabili.
\item Può essere nullo, anche se $[M]\neq 0$.
\end{itemize}
\end{frame}
\subsection{Principio di proporzionalità}
\begin{frame}{\secname}{\subsecname}
\onslide<2>{Lo dimostreremo con un'ipotesi aggiuntiva.}
\begin{block}{Teorema}
Sia $M$ una varietà Riemanniana chiusa\only<2>{ con curvatura non positiva}.\\
Allora il rapporto
\[
\frac{\norm{M}}{\Vol(M)}\onslide<2>{=\frac{1}{\norm{[\Vol_{\widetilde M}]^G_c}_\infty}}
\]
dipende solo dal tipo di isometria del rivestimento universale di $M$.
\end{block}
\end{frame}

\section{Applicazioni}
\subsection{Limitazione del grado}
\begin{frame}{\secname}{\subsecname}
\begin{block}{Proposizione}
Siano $M$, $N$ $n$-varietà chiuse orientate, $\map{f}{M}{N}$ una funzione continua di grado $d$. Allora
\[
|d|\cdot\norm{N}\le\norm{M}.
\]
\end{block}
\begin{block}{Corollario}
Sia $\map{f}{M}{M}$ di grado $d\ge 2$. Allora $\norm{M}=0$.
\end{block}
\end{frame}
\subsection{Varietà euclidee}
\begin{frame}{\secname}{\subsecname}
\begin{itemize}
\item L'$n$-toro $(S^1)^n$ ammette endomorfismi di grado arbitrariamente alto; di conseguenza, $\norm{(S^1)^n}=0$.
\item $(S^1)^n$ ammette una metrica piatta, con rivestimento universale isometrico a $\RR^n$.
\item Data una qualunque $n$-varietà chiusa euclidea $M$, vale
\[
\frac{\norm{M}}{\Vol(M)}=\frac{\norm{(S^1)^n}}{\Vol((S^1)^n)}=0
\]
da cui $\norm{M}=0$.
\end{itemize}
\end{frame}
\subsection{Varietà iperboliche}
\begin{frame}{\secname}{\subsecname}
\begin{block}{Teorema}
Sia $M$ una $n$-varietà chiusa iperbolica. Allora
\[
\norm{M}=\frac{\Vol(M)}{v_n},
\]
dove $v_n$ è il volume dell'$n$-simplesso ideale regolare in $\HH^n$.
\end{block}
\begin{block}{Corollario}
Una varietà chiusa $M$ non può ammettere contemporaneamente una metrica euclidea e una iperbolica.
\end{block}
\end{frame}
\subsection{Mappe fra superfici}
\begin{frame}{\secname}{\subsecname}
Sia $\Sigma_g$ la superficie chiusa orientabile di genere $g$.
\begin{itemize}
\item Per $g\le 1$ vale $\norm{\Sigma_g}=0$
\item Per $g\ge 2$, $\Sigma_g$ ammette una metrica iperbolica.
\[
\norm{\Sigma_g}=\frac{\Vol(\Sigma_g)}{v_2}=-\frac{2\pi\chi(\Sigma_g)}{\pi}=4g-4.
\]
\item Sia $\map{f}{\Sigma_{g_1}}{\Sigma_{g_2}}$, con $g_1\ge 1$, $g_2\ge 2$. Allora
\[
|\deg(f)|\le\frac{\norm{\Sigma_{g_1}}}{\norm{\Sigma_{g_2}}}=\frac{g_1-1}{g_2-1}.
\]
\end{itemize}
\end{frame}
\end{document}