\section{Coomologia continua}

\subsection{Definizioni}

Riportiamo la definizione di topologia compatta-aperta e ne ricordiamo alcune utili proprietà.

\begin{definition}
Siano $X$, $Y$ spazi topologici, $F(X,Y)$ l'insieme delle funzioni continue da $X$ in $Y$. La \defterm{topologia compatta-aperta} su $F(X,Y)$ è la topologia generata dai sottoinsiemi
\[
V(K,U)=\{f\in F(X,Y): f(K)\subs U\}
\]
al variare di $K\subs X$ compatto e di $U\subs Y$ aperto.
\end{definition}

\begin{lemma}\thlabel{compact-open-topology-lemma}
\begin{enumerate}
\item Siano $X$, $Y$, $Z$ spazi topologici, $\map{f}{Y}{Z}$, $\map{g}{X}{Y}$ funzioni continue. Allora le applicazioni
\begin{align*}
\map{(f\circ{-})}{F(X,Y)}{F(X,Z)},&&\map{({-}\circ g)}{F(Y,Z)}{F(X,Z)}
\end{align*}
sono continue.
\item Siano $X$, $Y$ spazi topologici con $X$ localmente compatto e Hausdorff. Allora l'applicazione di valutazione
\uMap{F(X,Y)\times X}{Y}{(f,x)}{f(x)}
è continua.
\end{enumerate}
\end{lemma}
Sia $M$ una $n$-varietà. Consideriamo sullo spazio $S_i(M)=F(\Delta^i,M)$ degli $i$-simplessi singolari la topologia compatta-aperta.

\begin{definition}
Una cocatena $\varphi\in C^i(M)$ si dice \defterm{continua} se la sua restrizione a $S_i(M)$ è continua.
\end{definition}

Osserviamo che se $\varphi\in C^i(M)$ è continua, allora anche $\varphi\circ d\in C^{i+1}(M)$ lo è (grazie al \thref{compact-open-topology-lemma}). Dunque le cocatene continue formano un sottocomplesso di $C^\bul(M)$, che indichiamo con $C^\bul_c(M)$. I moduli di coomologia relativi a questo complesso saranno denotati con $H^\bul_c(M)$. L'inclusione di complessi
\[
\map{i^\bul}{C^\bul_c(M)}{C^\bul(M)}
\]
induce mappe in coomologia
\[
\map{H^\bul(i^\bul)}{H^\bul_c(M)}{H^\bul(M)}.
\]
Il complesso $C^\bul_c(M)$ eredita per restrizione la seminorma $\ell^\infty$, che induce a sua volta una seminorma su $H^\bul_c(M)$.
In questa sezione ci domanderemo se la mappa $H^\bul(i^\bul)$ sia un isomorfismo isometrico, dando risposta affermativa nel caso in cui $M$ ammetta una metrica Riemanniana a curvatura non positiva.


\subsection{Cocatene continue e moduli iniettivi}

Sia $M$ una $n$-varietà chiusa, $\map{p}{\widetilde M}{M}$ il suo rivestimento universale. Fissiamo un'identificazione di $\Gamma=\pi_1(M)$ con il gruppo degli automorfismi di rivestimento di $p$. Ricordiamo che i moduli $C^i(\widetilde M)$ hanno una struttura naturale di $\RR[\Gamma]$-moduli.

\begin{proposition}\thlabel{cochains-injective-module}
Per ogni $i\ge 0$, il $\RR[\Gamma]$-modulo $C^i(\widetilde M)$ è iniettivo.
\end{proposition}
\begin{proof}
Sia $L_i(\widetilde M)$ un insieme di rappresentanti per l'azione di $\Gamma$ su $S_i(\widetilde M)$. In altre parole, per ogni $s\in S_i(\widetilde M)$ esistono unici $g_s\in\Gamma$, $\overline{s}\in L_i(\widetilde M)$ tali che $s=g_s\cdot\overline{s}$.

Siano $A$, $B$ due $\RR[\Gamma]$-moduli, $\map{\iota}{A}{B}$, $\map{\alpha}{A}{C^i(\widetilde M)}$ due applicazioni $\RR[\Gamma]$-lineari con $\iota$ iniettiva. Consideriamo un'inversa sinistra $\RR$-lineare $\map{\sigma}{B}{A}$ di $\iota$, ossia tale che $\sigma\circ\iota$ sia l'identità su $A$.
\begin{diagram}
0\rar&A\rar["\iota",swap]\dar{\alpha}&B\ar[dl,dashed,"\beta"]\lar["\sigma", bend right = 30,swap]\\
&C^i(\widetilde M)
\end{diagram}
Per ogni $b\in B$ definiamo la cocatena $\beta(b)\in C^i(\widetilde M)$ come l'unica applicazione $\RR$-lineare tale che per ogni $s\in S_i(\widetilde M)$ valga
\[
\beta(b)(s)=\alpha(\sigma(g_s^{-1}\cdot b))(\overline{s}).
\]
\begin{itemize}
\item\textbf{$\beta$ è $\Gamma$-lineare.} Sia $g\in\Gamma$. Osserviamo che $g^{-1}\cdot s=g^{-1}g_s\cdot\overline{s}$, dunque $g_{g^{-1}\cdot s}=g^{-1}g_s$ e $\overline{g^{-1}\cdot s}=\overline{s}$. Allora
\[
(g\cdot\beta(b))(s)=\beta(b)(g^{-1}\cdot s)=\alpha(\sigma(g_s^{-1}g\cdot b))(\overline{s})=\beta(g\cdot b)(s).
\]
\item\textbf{Vale $\beta\circ\iota=\alpha$.} Sia $a\in A$. Abbiamo
\[
\beta(\iota(a))(s)=\alpha(\sigma(g_s^{-1}\cdot(\iota(a))))(\overline{s})=\alpha(g_s^{-1}\cdot a)(\overline{s})=\alpha(a)(s).\qedhere
\]
\end{itemize}
\end{proof}

Osserviamo che per ogni $g\in\Gamma$ l'applicazione $\map{(g\cdot{-})}{S_i(\widetilde M)}{S_i(\widetilde M)}$ è continua (grazie al \thref{compact-open-topology-lemma}), dunque i moduli $C^i_c(\widetilde M)$ ereditano per restrizione una struttura di $\RR[\Gamma]$-moduli. Per mostrare un analogo della proposizione precedente per i moduli di cocatene continue, è necessario un lemma preliminare.

\begin{lemma}\thlabel{universal-covering-mysterious-function}
Esiste una funzione continua $\map{h_{\widetilde M}}{\widetilde M}{[0,1]}$ che soddisfa le seguenti proprietà:
\begin{enumerate}
\item per ogni $x\in\widetilde M$ esiste un intorno $W\subs \widetilde M$ di $x$ tale che l'insieme
\[
\{g\in\Gamma:g(W)\cap \supp h_{\widetilde M}\neq\emptyset\}
\]
è finito;
\item per ogni $x\in\widetilde M$ vale
\[
\sum_{g\in\Gamma}h_{\widetilde M}(g\cdot x)=1.
\]
\end{enumerate}
\end{lemma}

\begin{proposition}\thlabel{continuous-cochains-injective-module}
Per ogni $i\ge 0$, il $\RR[\Gamma]$-modulo $C^i_c(\widetilde M)$ è iniettivo.
\end{proposition}
\begin{proof}
Siano $A$, $B$ due $\RR[\Gamma]$-moduli, $\map{\iota}{A}{B}$, $\map{\alpha}{A}{C^i_c(\widetilde M)}$ due applicazioni $\RR[\Gamma]$-lineari con $\iota$ iniettiva. Consideriamo un'inversa sinistra $\RR$-lineare $\map{\sigma}{B}{A}$ di $\iota$.
\begin{diagram}
0\rar&A\rar["\iota",swap]\dar{\alpha}&B\ar[dl,dashed,"\beta"]\lar["\sigma", bend right = 30,swap]\\
&C^i_c(\widetilde M)
\end{diagram}
Sia $h_{\widetilde M}$ una funzione come nel \thref{universal-covering-mysterious-function}. Per ogni $b\in B$ definiamo la cocatena $\beta(b)\in C^i_c(\widetilde M)$ come l'unica applicazione $\RR$-lineare tale che per ogni $s\in S_i(\widetilde M)$ valga
\[
\beta(b)(s)=\sum_{g\in\Gamma}h_{\widetilde M}(g^{-1}(s(e_0)))\cdot \alpha(g(\sigma(g^{-1}\cdot b)))(s),
\]
dove $e_0,\ldots,e_i$ sono i vertici del simplesso standard $\Delta^i$. Osserviamo che, per le proprietà di $h_{\widetilde M}$, la somma su $g$ è in realtà una somma finita, dunque $\beta(b)(s)$ è ben definito.
\begin{itemize}
\item \textbf{$\beta(b)$ è una cocatena continua.} Per definizione di $h_{\widetilde M}$, per ogni $s\in S_i(\widetilde M)$ esiste un intorno $W\subs \widetilde M$ di $s(e_0)$ tale che
\[
\Gamma_s=\{g\in\Gamma:g^{-1}(W)\cap\supp h_{\widetilde M}\neq\emptyset\}
\]
è finito. Allora per ogni $s'\in V(\{e_0\},W)$ (che è un intorno di $s$ in $S_i(\widetilde M)$) vale
\[
\beta(b)(s')=\sum_{g\in\Gamma_s}h_{\widetilde M}(g^{-1}(s'(e_0)))\cdot \alpha(g(\sigma(g^{-1}\cdot b)))(s'),
\]
che è evidentemente continua in $s'$ (grazie al \thref{compact-open-topology-lemma}).
\item \textbf{$\beta$ è $\Gamma$-lineare.} Sia $g_0\in\Gamma$. Abbiamo
\begin{align*}
\beta(g_0\cdot b)(s)&=\sum_{g\in\Gamma}h_{\widetilde M}(g^{-1}(s(e_0)))\cdot \alpha(g(\sigma(g^{-1}g_0\cdot b)))(s)\\
&=\sum_{k\in\Gamma}h_{\widetilde M}(k^{-1}g_0^{-1}(s(e_0)))\cdot\alpha(g_0k(\sigma(k^{-1}\cdot b)))(s)\\
&=\sum_{k\in\Gamma}h_{\widetilde M}(k^{-1}(g_0^{-1}\cdot s)(e_0)))\cdot\alpha(k(\sigma(k^{-1}\cdot b)))(g_0^{-1}\cdot s)\\
&=\beta(b)(g_0^{-1}\cdot s)=(g_0\cdot\beta(b))(s).
\end{align*}
\item \textbf{Vale $\beta\circ\iota=\alpha$.} Sia $a\in A$. Abbiamo
\begin{align*}
\beta(\iota(a))(s)&=\sum_{g\in\Gamma}h_{\widetilde M}(g^{-1}(s(e_0)))\cdot \alpha(g(\sigma(g^{-1}\cdot\iota(a))))(s)\\
&=\sum_{g\in\Gamma}h_{\widetilde M}(g^{-1}(s(e_0)))\cdot \alpha(g(\sigma(\iota(g^{-1}\cdot a))))(s)\\
&=\sum_{g\in\Gamma}h_{\widetilde M}(g^{-1}(s(e_0)))\cdot \alpha(a)(s)=\alpha(a)(s).
\end{align*}
\end{itemize}
Abbiamo dunque mostrato che $C^i_c(\widetilde M)$ è un $\RR[\Gamma]$-modulo iniettivo.
\end{proof}

\subsection{Varietà con curvatura non positiva}
Nonostante si possa proseguire anche con ipotesi meno restrittive, per semplicità ci limiteremo a considerare, da qui alla fine della sezione, varietà chiuse $M$ che ammettano una metrica Riemanniana con curvatura non positiva.

In questo contesto, il teorema di Cartan-Hadamard garantisce che ogni coppia di punti $x,y\in\widetilde M$ siano collegati da un'unica geodetica; inoltre le parametrizzazioni a velocità costante delle geodetiche dipendono in modo liscio dagli estremi. Questo fatto permette di realizzare una procedura di raddrizzamento dei simplessi singolari.

\begin{definition}
Siano $x_0,\ldots,x_k$ punti di $\widetilde M$. Il \emph{simplesso dritto} di vertici $x_0,\ldots,x_k$ è il simplesso singolare $[x_0,\ldots,x_k]\in S_k(\widetilde M)$ definito induttivamente come segue.
\begin{itemize}
\item Se $k=0$, allora $[x_0]$ è lo $0$-simplesso avente immagine $x_0$.
\item Se $k>0$, allora $[x_0,\ldots,x_k]$ è univocamente determinato dalla seguente condizione: per ogni $z\in\Delta^{k-1}\subs\Delta^k$, la restrizione di $[x_0,\ldots,x_k]$ al segmento di estremi $z$ e $e_k$ è la parametrizzazione a velocità costante della geodetica che collega $[x_0,\ldots,x_{k-1}](z)$ a $x_k$.
\end{itemize}
\end{definition}

È facile vedere, grazie a Cartan-Hadamard, che la definizione è ben posta (ossia $[x_0,\ldots,x_k]$ è una funzione continua). Notiamo inoltre che, essendo gli elementi di $\Gamma$ isometrie di $\widetilde M$, vale l'identità
\[
g\circ[x_0,\ldots,x_k]=[g(x_0),\ldots,g(x_k)]
\]
per ogni $g\in\Gamma$.

È infine utile osservare che, essendo $M$ e $\widetilde M$ spazi metrici, la topologia compatta-aperta su $S_i(M)$ e $S_i(\widetilde M)$ coincide con quella della convergenza uniforme.

\subsection{Cocatene continue e risoluzioni iniettive}

\begin{proposition}\thlabel{cochains-resolution-R}
Il complesso di $\RR[\Gamma]$-moduli
\begin{diagram}
0\rar&\RR\rar{\epsilon}&C^0(\widetilde M)\rar{\delta^0}&C^1(\widetilde M)\rar{\delta^1}&\ldots\rar{\delta^{k-1}}&C^k(\widetilde M)\rar{\delta^k}&\ldots
\end{diagram}
è esatto, dove l'azione di $\RR[\Gamma]$ su $\RR$ è banale e $\epsilon(t)$ è la cocatena che vale $t$ su ogni $0$-simplesso.
\end{proposition}
\begin{proof}
Sappiamo che $\widetilde M$ è omeomorfo a $\RR^n$, dunque ha coomologia banale in tutte le dimensioni positive. Ciò implica che il complesso è esatto in $C^k(\widetilde M)$ per ogni $k\ge 1$. È immediato verificare che il complesso è esatto anche in $\RR$ e in $C^0(\widetilde M)$, da cui la tesi.
\end{proof}

\begin{proposition}\thlabel{continuous-cochains-resolution-R}
Il complesso di $\RR[\Gamma]$-moduli
\begin{diagram}
0\rar&\RR\rar{\epsilon}&C^0_c(\widetilde M)\rar{\delta^0}&C^1_c(\widetilde M)\rar{\delta^1}&\ldots\rar{\delta^{k-1}}&C^k_c(\widetilde M)\rar{\delta^k}&\ldots
\end{diagram}
è esatto.
\end{proposition}
\begin{proof}
Costruiremo un'omotopia $\RR$-lineare fra l'identità del complesso e la mappa nulla, mostrandone così l'esattezza. Fissiamo un punto base $x_0\in\widetilde M$. Definiamo per ogni $k\ge 0$ un operatore $\RR$-lineare $\map{T_k}{C_k(\widetilde M)}{C_{k+1}(\widetilde M)}$. Consideriamo l'applicazione
\Map{r}{\Delta^k}{\Delta^{k+1}}{t_0e_0+\ldots+t_ke_k}{t_0e_1+\ldots+t_ke_{k+1}}
che identifica $\Delta^k$ con la faccia di $\Delta^{k+1}$ opposta a $e_0$. Dato un simplesso singolare $s\in S_k(\widetilde M)$, definiamo $T_k(s)\in S_{k+1}(\widetilde M)$ come l'unico simplesso singolare che soddisfa la seguente condizione: per ogni $q\in\Delta^k$, la restrizione di $T_k(s)$ al segmento di estremi $e_0$ e $r(q)$ è la parametrizzazione a velocità costante della geodetica di $\widetilde M$ di estremi $x_0$ e $s(q)$. Grazie al teorema di Cartan-Hadamard, è facile verificare che $T_k(s)$ è ben definito e continuo, e che l'applicazione $\map{T_k}{S_k(\widetilde M)}{S_{k+1}(\widetilde M)}$ è continua. Estendendo $T_k$ per $\RR$-linearità, si ottiene una mappa $\map{T_k}{C_k(\widetilde M)}{C_{k+1}(\widetilde M)}$. Definiamo infine $\map{T_{-1}}{\RR}{C_0(\widetilde M)}$ come $T_{-1}(t)=tx_0$. Si verifica facilmente che $d_0\circ T_{-1}=\id_\RR$ e che $T_{k-1}\circ d_k+d_{k+1}\circ T_k=\id_{C_k(\widetilde M)}$ per ogni $k\ge 0$.

Definiamo ora per ogni $k\ge 0$ l'applicazione
\Map{h^k}{C^k_c(\widetilde M)}{C^{k-1}_c(\widetilde M)}{\varphi}{\varphi\circ T_{k-1}.}
Osserviamo che $h^k(\varphi)$ è effettivamente una cocatena continua, poiché la restrizione di $T_{k-1}$ a $S_{k-1}(\widetilde M)$ è continua. Dunque la famiglia di mappe $\{h^k\}_{k\ge 0}$ fornisce un'omotopia fra l'identità del complesso e l'applicazione nulla, da cui la tesi.
\end{proof}

Dato un $\RR[\Gamma]$-modulo $A$, una \defterm{risoluzione iniettiva} di $A$ è il dato di un complesso $(I^\bul,\delta^\bul)$ di $\RR[\Gamma]$-moduli iniettivi e di una mappa $\map{\epsilon}{A}{I^0}$ tali che la successione
\begin{diagram}
0\rar&A\rar{\epsilon}&I^0\rar{\delta^0}&I^1\rar{\delta^1}&\ldots\rar{\delta^{k-1}}&I^k\rar{\delta^k}&\ldots
\end{diagram}
sia esatta.

Ricordiamo un noto fatto di algebra omologica. \todo{Reference please?}

\begin{theorem}\thlabel{injective-resolutions}
Siano $A$, $B$ due $\RR[\Gamma]$-moduli. Siano $(I^\bul,\delta_I^\bul)$ una risoluzione iniettiva di $A$, $(J^\bul,\delta_J^\bul)$ una risoluzione iniettiva di $B$. Sia $\map{f}{A}{B}$ un morfismo di $\RR[\Gamma]$-moduli. Allora esiste un morfismo di complessi $\map{f^\bul}{I^\bul}{J^\bul}$ che \defterm{estende} $f$, ossia che fa commutare il diagramma
\begin{diagram}
0\rar&A\rar{\epsilon_I}\dar{f}&I^0\rar{\delta_I^0}\dar["f^0",dashed]&I^1\rar{\delta_I^1}\dar["f^1",dashed]&\ldots\\
0\rar&B\rar{\epsilon_J}&J^0\rar{\delta_J^0}&J^1\rar{\delta_J^1}&\ldots
\end{diagram}
Inoltre, ogni altro morfismo di complessi che estende $f$ è omotopo a $f^\bul$.
\end{theorem}

\subsection{Coomologia continua e coomologia singolare}

Esiste un funtore $(-)^\Gamma$ dalla categoria dei $\RR[\Gamma]$-moduli nella categoria degli $\RR$-moduli, che a ogni $\RR[\Gamma]$-modulo $A$ associa il sottospazio $A^\Gamma$ degli elementi $\Gamma$-invarianti, e agisce sui morfismi per restrizione.

\begin{lemma}\thlabel{continuous-gamma-invariant-cochains}
Il morfismo di complessi $\map{p^\bul}{C^\bul(M)}{C^\bul(\widetilde M)}$ induce per restrizione isomorfismi isometrici di complessi
\begin{align*}
\map{p^\bul}{C^\bul(M)}{C^\bul(\widetilde M)^\Gamma},&&\map{p^\bul}{C^\bul_c(M)}{C^\bul_c(\widetilde M)^\Gamma},
\end{align*}
i quali a loro volta inducono isomorfismi isometrici in coomologia
\begin{align*}
H^\bul(M)\iso H^\bul(C^\bul(M)^\Gamma),&&H^\bul_c(M)\iso H^\bul(C^\bul_c(\widetilde M)^\Gamma).
\end{align*}
\end{lemma}
\begin{proof}
È immediato verificare che $\map{p^\bul}{C^\bul(M)}{C^\bul(\widetilde M)^\Gamma}$ è un isomorfismo isometrico. Per concludere è sufficiente mostrare che una cocatena $\varphi\in C^k(M)$ è continua se e solo se lo è $\varphi\circ p_k\in C^k(\widetilde M)^\Gamma$. Tuttavia non è difficile vedere \todo{Reference please.} che la mappa $\map{(p\circ{-})}{S_k(\widetilde M)}{S_k(M)}$ è un rivestimento; in particolare è continua, suriettiva e aperta, da cui la tesi.
\end{proof}

Possiamo infine dimostrare il risultato principale di questa sezione.

\begin{theorem}\thlabel{continuous-cohomology-isomorphism}
Sia $M$ una varietà Riemanniana chiusa con curvatura non positiva. Allora l'applicazione
\[
\map{H^\bul(i^\bul)}{H^\bul_c(M)}{H^\bul(M)}
\]
è un isomorfismo isometrico.
\end{theorem}
\begin{proof}
In virtù della \thref{cochains-injective-module} e della \thref{cochains-resolution-R}, il complesso
\begin{diagram}
0\rar&\RR\rar{\epsilon}&C^0(\widetilde M)\rar{\delta^0}&C^1(\widetilde M)\rar{\delta^1}&\ldots
\end{diagram}
fornisce una risoluzione iniettiva del $\RR[\Gamma]$-modulo banale $\RR$. Analogamente, grazie alla \thref{continuous-cochains-injective-module} e alla \thref{continuous-cochains-resolution-R}, anche il complesso
\begin{diagram}
0\rar&\RR\rar{\epsilon}&C^0_c(\widetilde M)\rar{\delta^0}&C^1_c(\widetilde M)\rar{\delta^1}&\ldots
\end{diagram}
descrive una risoluzione iniettiva di $\RR$. Abbiamo inoltre l'inclusione di complessi $\map{j^\bul}{C^\bul_c(\widetilde M)}{C^\bul(\widetilde M)}$, che è un morfismo $1$-Lipschitz che estende l'identità di $\RR$.
\begin{diagram}
0\rar&\RR\rar{\epsilon}\dar{\id_\RR}&C^0_c(\widetilde M)\rar{\delta^0}\dar{j^0}&C^1_c(\widetilde M)\rar{\delta^1}\dar{j^1}&\ldots\\
0\rar&\RR\rar{\epsilon}&C^0(\widetilde M)\rar{\delta^0}&C^1(\widetilde M)\rar{\delta^1}&\ldots
\end{diagram}

Esibiremo ora un morfismo di complessi $1$-Lipschitz $\map{\theta^\bul}{C^\bul(\widetilde M)}{C^\bul_c(\widetilde M)}$ che estenda l'identità di $\RR$.

\begin{diagram}
0\rar&\RR\rar{\epsilon}\dar{\id_\RR}&C^0(\widetilde M)\rar{\delta^0}\dar{\theta^0}&C^1(\widetilde M)\rar{\delta^1}\dar{\theta^1}&\ldots\\
0\rar&\RR\rar{\epsilon}&C^0_c(\widetilde M)\rar{\delta^0}&C^1_c(\widetilde M)\rar{\delta^1}&\ldots
\end{diagram}

Fissiamo un punto base $x_0\in\widetilde M$. Per ogni $\varphi\in C^k(\widetilde M)$ e per ogni $s\in S_k(\widetilde M)$ definiamo
\[
\theta^k(\varphi)(s)=\sum_{(g_0,\ldots,g_k)\in\Gamma^{k+1}}h_{\widetilde M}(g_0^{-1}(s(e_0)))\cdots h_{\widetilde M}(g_k^{-1}(s(e_k)))\cdot\varphi([g_0(x_0),\ldots,g_k(x_0)]),
\]
dove $\map{h_{\widetilde M}}{\widetilde M}{[0,1]}$ è data dal \thref{universal-covering-mysterious-function}. Grazie alle proprietà di $h_{\widetilde M}$ è facile verificare che $\theta^k(\varphi)$ (una volta estesa per $\RR$-linearità) definisce un elemento di $C^k_c(\widetilde M)$, e che $\theta^\bul$ risulta essere un morfismo $1$-Lipschitz di complessi di $\RR[\Gamma]$-moduli che estende l'identità di $\RR$.

Osserviamo che
\[
\map{\theta^\bul\circ j^\bul}{C^\bul_c(\widetilde M)}{C^\bul_c(\widetilde M)}
\]
è un morfismo di complessi che estende l'identità di $\RR$. Dal \thref{injective-resolutions} segue che tale morfismo è omotopo all'identità. Questa proprietà si mantiene applicando il funtore $(-)^\Gamma$, dunque otteniamo che
\[
H^\bul(\theta^\bul)\circ H^\bul(j^\bul)=\id_{H^\bul(C^\bul(\widetilde M)^\Gamma)}.
\]
Allo stesso modo,
\[
H^\bul(j^\bul)\circ H^\bul(\theta^\bul)=\id_{H^\bul(C^\bul_c(\widetilde M)^\Gamma)}.
\]
Essendo $H^\bul(j^\bul)$ e $H^\bul(\theta^\bul)$ morfismi $1$-Lipschitz, segue in particolare che
\[
\map{H^\bul(j^\bul)}{H^\bul(C^\bul_c(\widetilde M)^\Gamma)}{H^\bul(C^\bul(\widetilde M)^\Gamma)}
\]
è un isomorfismo isometrico.

Dal diagramma commutativo di complessi
\begin{diagram}
C^\bul_c(M)\rar["p^\bul","\iso"']\dar{i^\bul}&C^\bul_c(\widetilde M)^\Gamma\dar{j^\bul}\\
C^\bul(M)\rar["p^\bul","\iso"']&C^\bul(\widetilde M)^\Gamma
\end{diagram}
(in cui le frecce orizzontali sono isomorfismi isometrici per il \thref{continuous-gamma-invariant-cochains}) segue che anche $H^\bul(i^\bul)$ è un isomorfismo isometrico.
\end{proof}