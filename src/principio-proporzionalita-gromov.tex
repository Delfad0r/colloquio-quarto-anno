\section{Principio di proporzionalità di Gromov}

\subsection{Mappa di restrizione}
Utilizziamo le notazioni della sezione precedente, continuando a supporre che $M$ sia una varietà Riemanniana chiusa con curvatura non positiva. Sia $G$ il gruppo delle isometrie di $\widetilde M$ che preservano l'orientazione. È ben noto che $G$ ammette una struttura di gruppo di Lie che induce la topologia compatta-aperta. Di conseguenza esiste una misura di Borel regolare invariante a sinistra su $G$ (\defterm{misura di Haar}), unica a meno di riscalamento.

Poiché $\Gamma$ è un sottogruppo discreto di $G$ e $M\iso\widetilde M/\Gamma$ è compatta, esiste un insieme misurabile $F\subs G$ relativamente compatto tale che $\{\gamma\cdot F\}_{\gamma\in\Gamma}$ definisca una partizione localmente finita di $G$. In particolare, $\Gamma$ è cocompatto in $G$, pertanto la misura di Haar è anche invariante a destra. D'ora in poi supporremo che tale misura sia riscalata in modo che $F$ abbia misura $1$.
\todo{Reference please.}

\begin{definition}
Indichiamo con $C^\bul_c(\widetilde M)^G$ il complesso delle cocatene continue $G$-invarianti. L'inclusione di complessi $\umap{C^\bul_c(\widetilde M)^G}{C^\bul_c(\widetilde M)^\Gamma}$ induce una mappa in coomologia
\[
\map{\res^\bul}{H^\bul(C^\bul_c(\widetilde M)^G)}{H^\bul(C^\bul_c(\widetilde M)^\Gamma)}
\]
detta \defterm{mappa di restrizione}.
\end{definition}

Osserviamo che, considerando su $H^\bul(C^\bul_c(\widetilde M)^G)$ e $H^\bul(C^\bul_c(\widetilde M)^\Gamma)$ le seminorme indotte rispettivamente da $C^\bul_c(\widetilde M)^G$ e $C^\bul_c(\widetilde M)^\Gamma$, la mappa di restrizione risulta $1$-Lipschitz.

Ci proponiamo ora di costruire un'inversa sinistra $1$-Lipschitz di $\res^\bul$. Indichiamo con $\mu_G$ la misura di Haar su $G$. Per ogni $\varphi\in C^i_c(\widetilde M)$ e per ogni $s\in S_i(\widetilde M)$ definiamo
\[
\trans^i(\varphi)(s)=\int_F\varphi(g\cdot s)d\mu_G(g).
\]
Si tratta di una buona definizione, poiché $\varphi({-}\cdot s)$ è una funzione continua da $G$ in $\RR$ e $F$ è relativamente compatto. Estendendo $\trans^i(\varphi)$ per linearità, otteniamo un elemento di $C^i(\widetilde M)$.

\begin{proposition}\thlabel{transfer-map-properties}
Per ogni $\varphi\in C^i_c(\widetilde M)$ valgono le seguenti proprietà.
\begin{enumerate}
\item La cocatena $\trans^i(\varphi)$ è continua.
\item Vale $\trans^{i+1}(\varphi\circ d^{i+1})=\trans^i(\varphi)\circ d^{i+1}$.
\item Se $\varphi$ è $\Gamma$-invariante, allora $\trans^i(\varphi)$ è $G$-invariante.
\item Se $\varphi$ è $G$-invariante, allora $\trans^i(\varphi)=\varphi$.
\end{enumerate}
\end{proposition}
\begin{proof}\leavevmode
\begin{enumerate}
\item Osserviamo innanzitutto che la topologia compatta-aperta su $S_i(\widetilde M)$ è indotta dalla distanza
\[
\dist(s,s')=\sup\{\dist_{\widetilde M}(s(x),s'(x)):x\in\Delta^i\}.
\]
Sia $s_0\in S_i(\widetilde M)$, e sia $\epsilon>0$. Poiché $\overline{F}$ è compatto in $G$, dal \thref{compact-open-topology-lemma} si ottiene immediatamente che $\overline{F}\cdot s_0$ è compatto in $S_i(\widetilde M)$. Dalla continuità di $\varphi$ segue facilmente l'esistenza di un $\eta>0$ tale che per ogni $s\in\overline{F}\cdot s_0$ e per ogni $s'\in S_i(\widetilde M)$ con $\dist(s,s')<\eta$ valga $|\varphi(s)-\varphi(s')|\le\epsilon$. Sia dunque $s\in S_i(\widetilde M)$ tale che $\dist(s_0,s)<\eta$. Poiché $G$ agisce su $S_i(\widetilde M)$ in modo isometrico, allora anche $\dist(g\cdot s_0,g\cdot s)<\eta$ per ogni $g\in G$. Ma allora
\[
|\trans^i(\varphi)(s)-\trans^i(\varphi)(s_0)|\le\int_F|\varphi(g\cdot s)-\varphi(g\cdot s')|d\mu_G(g)\le\epsilon\mu_G(F)=\epsilon
\]
dunque $\trans^i(\varphi)$ è continua.
\item Sia $s\in S_{i+1}(\widetilde M)$, e siano $a_0,\ldots,a_{i+1}\in\RR$, $s_0,\ldots,s_{i+1}\in S_i(\widetilde M)$ tali che
\[
d^{i+1}(s)=\sum_{j=0}^{i+1}a_js_j.
\]
Osserviamo che
\[
d^{i+1}(g\cdot s)=\sum_{j=0}^ra_j(g\cdot s_j),
\]
per ogni $g\in G$, da cui
\begin{align*}
\trans^{i+1}(\varphi\circ d^{i+1})(s)&=\int_F\varphi(d^{i+1}(g\cdot s))d\mu_G(g)\\
&=\sum_{j=0}^{i+1}a_j\int_F\varphi(g\cdot s_j)d\mu_G(g)\\
&=\sum_{j=0}^{i+1}a_j\trans^i(\varphi)(s_j)\\
&=\trans^i(\varphi)\left(\sum_{j=0}^{i+1}a_js_j\right)=\trans^i(\varphi)(d^{i+1}s).
\end{align*}
\item Fissiamo $\varphi\in C^i_c(\widetilde M)$, $s\in S_i(\widetilde M)$, $g_0\in G$. Poiché $F$ è relativamente compatto, lo sono anche $F\cdot g_0$ e $F\cdot g_0^{-1}$, dunque esistono un numero finito di elementi $\gamma_1,\ldots,\gamma_r\in\Gamma$ tali che
\begin{align*}
F\cdot g_0\subs\bigsqcup_{j=1}^r\gamma_j\cdot F&&\text{e}&&F\cdot g_0^{-1}\subs\bigsqcup_{j=1}^r\gamma_j^{-1}\cdot F.
\end{align*}
Posto $F_j=(\gamma_j^{-1}\cdot F\cdot g_0)\cap F$ si ottiene immediatamente che
\begin{align*}
F=\bigsqcup_{j=1}^rF_j&&\text{e}&&F\cdot g_0=\bigsqcup_{j=1}^r\gamma_j\cdot F_j.
\end{align*}
Sfruttando il fatto che $\mu_G$ è invariante a destra e a sinistra e che $\varphi$ è $\Gamma$-invariante si ottiene
\begin{align*}
\trans^i(\varphi)(g_0\cdot s)&=\int_F\varphi(gg_0\cdot s)d\mu_G(g)\\
&=\int_{F\cdot g_0}\varphi(g\cdot s)d\mu_G(g)\\
&=\sum_{j=1}^r\int_{\gamma_j\cdot F_j}\varphi(g\cdot s)d\mu_G(g)\\
&=\sum_{j=1}^r\int_{F_j}\varphi(\gamma_jg\cdot s)d\mu_G(g)\\
&=\sum_{j=1}^r\int_{F_j}\varphi(g\cdot s)d\mu_G(g)\\
&=\int_F\varphi(g\cdot s)d\mu_G(g)=\trans(\varphi)(s).
\end{align*}
\item Se $\varphi$ è $G$-invariante segue immediatamente dalla definizione che $\trans^i(\varphi)=\varphi$.\qedhere
\end{enumerate}
\end{proof}

\begin{corollary}\thlabel{restriction-map-isometric-embedding}
La mappa di restrizione
\[
\map{\res^\bul}{H^\bul(C^\bul_c(\widetilde M)^G)}{H^\bul(C^\bul_c(\widetilde M)^\Gamma)}
\]
è un'immersione isometrica.
\end{corollary}
\begin{proof}
Dalla \thref{transfer-map-properties} segue immediatamente che
\[
\map{\trans^\bul}{C^\bul_c(\widetilde M)^\Gamma}{C^\bul_c(\widetilde M)^G}
\]
è un morfismo di complessi ben definito la cui restrizione a $C^\bul_c(\widetilde M)^G$ è l'identità. Poiché $\trans^\bul$ è evidentemente $1$-Lipschitz, guardando la corrispondente mappa in coomologia si ottiene che
\[
\map{H^\bul(\trans^\bul)}{H^\bul(C^\bul_c(\widetilde M)^\Gamma)}{H^\bul(C^\bul_c(\widetilde M)^G)}
\]
è una mappa $1$-Lipschitz tale che $H^\bul(\trans^\bul)\circ \res^\bul$ sia l'identità. Questo conclude la dimostrazione.
\end{proof}

\subsection{Cociclo volume}
Se $X$ è una varietà Riemanniana, denotiamo con $\prepost[_s]{S}{_k}(X)$ lo spazio dei $k$-simplessi \defterm{lisci} di $X$ (ossia l'insieme delle funzioni lisce da $\Delta^k$ in $X$) munito della topologia $C^1$.

\begin{proposition}\thlabel{straight-simplices-are-smooth}
Per ogni $x_0,\ldots,x_k\in\widetilde M$, il simplesso dritto $[x_0,\ldots,x_k]$ è liscio. Inoltre l'applicazione
\uMap{\widetilde M\times\ldots\times\widetilde M}{\prepost[_s]{S}{_k}(\widetilde M)}{(x_0,\ldots,x_k)}{[x_0,\ldots,x_k]}
è continua.
\end{proposition}
\begin{proof}

\end{proof}

Per ogni $s\in S_k(\widetilde M)$ definiamo $\tstr_k(s)=[s(e_0),\ldots,s(e_k)]$. Estendendo $\tstr_k$ per $\RR$-linearità otteniamo un'applicazione $\map{\tstr_k}{S_k(\widetilde M)}{S_k(\widetilde M)}$.

\begin{proposition}\thlabel{straightening-map-properties}
L'applicazione $\map{\tstr_k}{S_k(\widetilde M)}{S_k(\widetilde M)}$ soddisfa le seguenti proprietà.
\begin{enumerate}
\item $d_{k+1}\circ\tstr_{k+1}=\tstr_{k+1}\circ d_k$ per ogni $k\ge 0$ (ossia $\map{\tstr_\bul}{C_\bul(\widetilde M)}{C_\bul(\widetilde M)}$ è un morfismo di complessi).
\item $\tstr_k(\gamma\circ s)=\gamma\circ\tstr_k(s)$ per ogni $k\ge 0$, $\gamma\in\Gamma$, $s\in S_k(\widetilde M)$.
\item $\map{\tstr_\bul}{C_\bul(\widetilde M)}{C_\bul(\widetilde M)}$ è omotopa all'identità mediante un'omotopia $\Gamma$-equivariante che manda simplessi lisci in una somma finita di simplessi lisci.
\end{enumerate}
\end{proposition}
\begin{proof}
\leavevmode
\begin{enumerate}
\item È immediato verificare che la $i$-esima faccia di $[x_0,\ldots,x_k]$ è $[x_0,\ldots,\widehat{x_i},\ldots,x_k]$, da cui la tesi.
\item Poiché le isometrie preservano le geodetiche, si ha
\[
\gamma\circ[x_0,\ldots,x_k]=[\gamma(x_0),\ldots,\gamma(x_k)],
\]
da cui la tesi.
\item Dato un simplesso singolare $s\in S_k(\widetilde M)$ definiamo l'applicazione $\map{F}{\Delta^k\times[0,1]}{\widetilde M}$ in modo che per ogni $x\in\Delta^k$ la mappa $\map{F(x,-)}{[0,1]}{\widetilde M}$ sia la parametrizzazione a velocità costante della geodetica che congiunge $s(x)$ con $\tstr_k(s)(x)$. Definiamo poi $T_k(s)=F_\bul(c)$, dove $c\in C_{k+1}(\Delta^k\times[0,1])$ è la triangolazione standard di $\Delta^k\times[0,1]$. È immediato verificare che $d_{k+1}\circ T_k+T_{k-1}\circ d_k=\id-\tstr_k$, mentre la $\Gamma$-equivarianza di $T_k$ segue dal fatto che le isometrie preservano le geodetiche.\qedhere
\end{enumerate}
\end{proof}

Come conseguenza otteniamo un morfismo di complessi $\map{\str_\bul}{C_\bul(M)}{C_\bul(M)}$ omotopo all'identità. Inoltre, dalla \thref{straight-simplices-are-smooth}, $\str_k(s)$ è un simplesso liscio di $M$ per ogni $s\in S_k(M)$, e le restrizioni $\map{\str_k}{S_k(M)}{\prepost[_s]{S}{_k}(M)}$ sono continue.

Supponiamo ora che $M$ sia orientata, e sia $n$ la dimensione di $M$. Denotiamo con $\omega_M\in\Omega^n(M)$ la forma volume di $M$.
\begin{definition}
Per ogni $s\in S_n(M)$ definiamo
\[
\Vol_M(s)=\int_{\str_n(s)}\omega_M.
\]
Estendendo per linearità, otteniamo una cocatena $\Vol_M\in C^n(M)$, detta \defterm{cocatena volume}.
\end{definition}
Poiché $\map{\str_n}{S_n(M)}{\prepost[_s]{S}{_n}(M)}$ è continua e l'integrazione è continua rispetto alla topologia $C^1$, otteniamo che la cocatena volume è continua. Osserviamo inoltre che per ogni $s\in S_{n+1}(M)$ vale
\[
\Vol_M(d(s))=\int_{\str_n(d(s))}\omega_M=\int_{d\str_{n+1}(s)}\omega_M=\int_{\str_{n+1}(s)}d\omega_M=0,
\]
dove abbiamo utilizzato il fatto che $\str_\bul$ è un morfismo di complessi e il teorema di Stokes. Pertanto $\Vol_M$ è un cociclo, e definisce classi $[\Vol_M]\in H^n(M)$, $[\Vol_M]_c\in H^n_c(M)$ in coomologia.

\begin{lemma}\thlabel{volume-and-fundamental-coclass}
Vale $[\Vol_M]=\Vol(M)\cdot[M]^*$.
\end{lemma}
\begin{proof}
Osserviamo innanzitutto che $[\Vol_M]=\langle[\Vol_M],[M]\rangle\cdot[M]^*$. È ben noto che la classe fondamentale di $M$ ammette un rappresentante della forma $c=\sum s_i$, dove gli $s_i$ sono gli $n$-simplessi lisci e orientati positivamente di una triangolazione di $M$. Per la \thref{straightening-map-properties}, $c-\str_n(c)$ è bordo di una catena di simplessi lisci. Pertanto
\[
\langle[\Vol_M],[M]\rangle=\Vol_M(c)=\int_{\str_n(c)}\omega_M=\int_{c}\omega_M=\Vol(M),
\]
dove abbiamo usato il teorema di Stokes per dedurre che
\[
\int_{c-\str_n(c)}\omega_M=0.\qedhere
\]
\end{proof}

\subsection{Principio di proporzionalità}

Consideriamo l'immagine di $\Vol_M$ mediante l'identificazione isometrica $C^n_c(M)\iso C^n_c(\widetilde M)^\Gamma$ indotta da $p^\bul$: si tratta del cociclo $\map{\Vol_{\widetilde M}}{C_n(\widetilde M)}{\RR}$ tale che per ogni simplesso $s\in S_n(\widetilde M)$ valga
\[
\Vol_{\widetilde M}(s)=\int_{\str_n(p\circ s)}\omega_M=\int_{p\circ\tstr_n(s)}\omega_M=\int_{\tstr_n(s)}\omega_{\widetilde M},
\]
dove $\omega_{\widetilde M}$ è la forma volume di $\widetilde M$. Osserviamo che $\Vol_{\widetilde M}$ è una cocatena $G$-invariante, dunque definisce una classe $[\Vol_{\widetilde M}]^G_c\in H^n(C^\bul_c(\widetilde M))$ tale che l'immagine di $\res^n([\Vol_{\widetilde M}]^G_c)$ mediante l'identificazione isometrica $H^n(C^\bul_c(\widetilde M)^\Gamma)\iso H^n_c(M)$ sia proprio $[\Vol_M]_c$.

Possiamo ora enunciare e dimostrare il risultato principale di questa sezione.

\begin{theorem}\thlabel{gromov-proportionality-principle}
Sia $M$ una varietà Riemanniana chiusa con curvatura non positiva. Allora
\[
\norm{M}=\frac{\Vol(M)}{\norm{[\Vol_{\widetilde M}]^G_c}_\infty}.
\]
\end{theorem}
\begin{proof}
Per come è definito il volume simpliciale per varietà non orientabili, possiamo supporre che $M$ sia orientata. Cominciamo osservando che tutte le mappe nel seguente diagramma sono isomorfismi o immersioni isometriche (rispettivamente per la \thref{continuous-cohomology-isomorphism}, il \thref{continuous-gamma-invariant-cochains} e il \thref{restriction-map-isometric-embedding}).
\begin{diagram}
H^n(M)&H^n_c(M)\lar[swap,"H^n(i^\bul)","\iso"']\rar["H^n(p^\bul)","\iso"']&H^n(C^\bul_c(\widetilde M)^\Gamma)&H^n(C^\bul_c(\widetilde M)^G)\lar[swap,"\res^n"]
\end{diagram}
Inoltre, a $[\Vol_M]\in H^n(M)$ a sinistra corrisponde $[\Vol_{\widetilde M}]^G_c\in H^n(C^\bul_c(\widetilde M)^G)$ a destra; in particolare, $\norm{[\Vol_M]}_\infty=\norm{[\Vol_{\widetilde M}]^G_c}_\infty$. Allora la tesi segue immediatamente dalla \thref{simplicial-volume-duality} e dal \thref{volume-and-fundamental-coclass}:
\[
\norm{M}=\frac{1}{\norm{[M]^*}_\infty}=\frac{\Vol(M)}{\norm{[\Vol_M]}_\infty}=\frac{\Vol(M)}{\norm{[\Vol_{\widetilde M}]^G_c}_\infty}.\qedhere
\]
\end{proof}

In particolare, abbiamo il seguente.

\begin{corollary}\thlabel{gromov-proportionality-corollary}
Nelle ipotesi del teorema precedente, il rapporto $\norm{M}/\Vol(M)$ dipende solo dalla classe di isometria del rivestimento universale $\widetilde M$.
\end{corollary}