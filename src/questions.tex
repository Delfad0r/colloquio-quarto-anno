\section*{Domande}
Domande, vagamente in ordine di importanza.
\begin{itemize}
\item Conosce per caso qualche dimostrazione del fatto che le varietà chiuse semplicemente connesse hanno volume simpliciale nullo che non fa uso della coomologia limitata? Sembra un risultato interessante e mi piacerebbe includerlo nella mia trattazione. Tuttavia, se si rivelasse necessaria l'introduzione della coomologia limitata e del teorema per cui essa è isomorfa alla coomologia del gruppo fondamentale in piena generalità, credo che alla fine l'esposizione potrebbe risultare troppo dispersiva.
\item Sicuramente alla fine dell'esposizione la commissione vorrà chiedermi quali applicazioni abbia il principio di proporzionalità. Un fatto senza dubbio interessante è il \thref{flat-xor-hyperbolic}, che credo sia assolutamente non banale da dimostrare per altre vie: mi sbaglio? Ha in mente altre applicazioni che potrei esibire per soddisfare il desiderio di concretezza della commissione?
\item Giusto per essere sicuro di aver capito bene, nella dimostrazione della \thref{simplicial-volume-hyperbolic-manifold}, il motivo per cui $g\circ\del_is_\ell=\del_is_\ell\circ\overline{\sigma}$ è che entrambi i membri sono ``lineari'' (più precisamente, la parametrizzazione baricentrica si ottiene prendendo la combinazione convessa dei vertici in $\RR^{n,1}$ e poi proiettando sull'iperboloide, e le isometrie in questo modello sono lineari)?
\item Nel libro (precisamente a pagina 111) non riesco bene a seguire la dimostrazione della Proposizione 8.8. Probabilmente sono io che mi perdo in sciocchezze insiemistiche, ma come si deduce l'implicazione
\[
F=\bigsqcup_{i=1}^r F_i\implies F\cdot g_0=\bigsqcup_{i=1}^r \gamma_i\cdot F_i?
\]
Non capendolo (ma di nuovo, probabilmente sono io che mi perdo qualcosa) ho dovuto modificare lievemente la dimostrazione (\thref{transfer-map-properties}).
\item Ci sono altre domande sparse nel testo (arancioni, a fianco), ma sono di secondaria importanza, potremo parlarne più avanti.
\end{itemize}