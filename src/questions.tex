\section*{Domande}
Domande, vagamente in ordine di importanza.
\begin{itemize}
\item Dopo aver scritto tutta la parte sulla dimostrazione del principio di proporzionalità di Gromov, mi sono accorto che di fatto non serve mai parlare di coomologia limitata, nonostante compaia in alcuni enunciati presi direttamente dal libro (\thref{continuous-cochains-relatively-injective-module}, \thref{continuous-cochains-strong-resolution-R}, \thref{continuous-gamma-invariant-cochains}, \thref{continuous-cohomology-isomorphism}). Ai fini della mia (ridotta) trattazione, è sufficiente dare una nozione di \emph{seminorma} (eventualmente infinita) sui complessi di (co)catene e sui moduli di (co)omologia, come ho fatto nella prima sezione. Sebbene da una lettura anche superficiale del libro sia evidente che i concetti di volume simpliciale e di coomologia limitata sono strettamente legati, stavo pensando che forse per questa trattazione sia meglio escludere il secondo, che rischierebbe altrimenti di sembrare ``tirato fuori dal nulla" e poco pertinente per gli obiettivi che la trattazione si pone.
\item Nello spirito della domanda precedente, siccome in ultima analisi parlerò solo di $R[\Gamma]$-moduli per $R=\RR$, le nozioni di \emph{relativa iniettività} (e di conseguenza di \emph{risoluzione forte}) non sono necessarie. Pensavo anche in questo caso di tralasciarle, parlando solo di risoluzioni iniettive, potendo così invocare ben noti risultati di algebra omologica per la dimostrazione del \thref{continuous-cohomology-isomorphism}. In ogni caso (mi corregga se sbaglio) mi pare che il modo migliore di dimostrare che il complesso $C^\bul_c(\widetilde M)$ fornisce una risoluzione di $\RR$ sia comunque di esibire un'omotopia fra $0$ e l'identità.
\item Nella dimostrazione della \thref{simplicial-volume-hyperbolic-manifold}, ho azzardato un argomento che evita la parametrizzazione baricentrica. Probabilmente non funzionerà, vista la mia ancora scarsa dimestichezza con la geometria iperbolica, ma vorrei chiedere il suo parere.
\item Nel libro (precisamente a pagina 111) non riesco bene a seguire la dimostrazione della Proposizione 8.8. Probabilmente sono io che mi perdo in sciocchezze insiemistiche, ma come si deduce l'implicazione
\[
F=\bigsqcup_{i=1}^r F_i\implies F\cdot g_0=\bigsqcup_{i=1}^r \gamma_i\cdot F_i?
\]
Non capendolo (ma di nuovo, probabilmente sono io che mi perdo qualcosa) ho dovuto modificare lievemente la dimostrazione (\thref{transfer-map-properties}).
\item Ci sono altre domande sparse nel testo (arancioni, a fianco), ma sono di secondaria importanza, potremo parlarne più avanti.
\end{itemize}