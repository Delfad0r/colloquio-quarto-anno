\section{Volume simpliciale}

\subsection{Complessi seminormati}

\begin{definition}
Una \defterm{seminorma} su un $\RR$-spazio vettoriale $V$ è una funzione $\map{\norm{-}}{V}{[0,\infty]}$ tale che:
\begin{enumerate}
\item $\norm{\lambda\cdot v}=|\lambda|\cdot\norm{v}$ per ogni $\lambda\in\RR$, $v\in V$ (per convenzione, $0\cdot\infty=0$);
\item $\norm{v+w}\le\norm{v}+\norm{w}$ per ogni $v,w\in V$.
\end{enumerate}
\end{definition}
Osserviamo che, contrariamente all'uso comune, ammettiamo anche $\infty$ come possibile valore assunto dalla seminorma. Una \defterm{norma} su $V$ è una seminorma che soddisfa inoltre le seguenti proprietà:
\begin{enumerate}
\setcounter{enumi}{2}
\item $\norm{v}<\infty$ per ogni $v\in V$;
\item se $v\in V\setminus\{0\}$ allora $\norm{v}>0$.
\end{enumerate}

Dati una seminorma $\norm{-}$ su $V$ e un sottospazio vettoriale $W\subs V$, il quoziente $V/W$ eredita una seminorma naturale così definita: per ogni $[v]\in V/W$,
\[
\norm{[v]}=\inf\{\norm{v'}:v'\in V,[v]=[v']\}.
\]

Se $V$ e $W$ sono $\RR$-spazi vettoriali seminormati, un'applicazione lineare $\map{f}{V}{W}$ si dice:
\begin{itemize}
\item \defterm{$L$-Lipschitz} se $\norm{f(v)}\le L\cdot\norm{v}$ per ogni $v\in V$;
\item \defterm{isometrica} se $\norm{f(v)}=\norm{v}$ per ogni $v\in V$.
\end{itemize}
Un'\defterm{immersione isometrica} è un'isometria iniettiva fra spazi seminormati (osserviamo che un'isometria non è necessariamente iniettiva).

Dati due spazi vettoriali seminormati $V_1$, $V_2$, un'applicazione lineare $L$-Lipschitz $\map{f}{V_1}{V_2}$ e due sottospazi $W_1\subs V_1$, $W_2\subs V_2$ tali che $f(W_1)\subs W_2$, si verifica facilmente che l'applicazione indotta $\map{\overline{f}}{V_1/W_1}{V_2/W_2}$ è ancora $L$-Lipschitz. Al contrario, la proprietà di essere isometrica non si conserva per passaggio al quoziente.

\begin{definition}
Un \defterm{complesso seminormato} è un complesso $(C_\bul,d_\bul)$ di $\RR$-spazi vettoriali in cui ogni $C_i$ è dotato di una seminorma.
\end{definition}

Dalla discussione precedente deriva che gli $\RR$-spazi vettoriali $H_i(C_\bul)$ ereditano in modo naturale una seminorma. Inoltre, un morfismo di complessi $L$-Lipschitz $\map{f_\bul}{C_\bul}{C'_\bul}$ induce applicazioni $L$-Lipschitz $\map{H_\bul(f_\bul)}{H_\bul(C_\bul)}{H_\bul(C'_\bul)}$ in omologia.

\subsection{Seminorme singolari e prodotto di Kronecker}

Sia $X$ uno spazio topologico. Poiché in questa trattazione ci occuperemo quasi esclusivamente di (co)omologia a coefficienti reali, laddove non specificato i moduli di (co)catene e di (co)omologia saranno da intendersi a coefficienti reali. Muniamo il complesso delle catene singolari $(C_\bul(X),d)$ della norma $\ell^1$, definita come segue: per ogni $\sum_{i=1}^ka_is_i\in C_n(X)$ vale
\[
\norm{\sum_{i=1}^ka_is_i}_1=\sum_{i=1}^k|a_i|.
\]
Definiamo inoltre la seminorma $\ell^\infty$ sul complesso delle cocatene singolari $(C^\bul(X),\delta)$: data una cocatena $\varphi\in C^n(X)$, la sua seminorma $\ell^\infty$ è
\[
\norm{\varphi}_\infty=\sup\{|\varphi(c)|:c\in C_n(X),\norm{c}_1\le 1\}.
\]
Di conseguenza i moduli di omologia e coomologia di $X$ ereditano, rispettivamente, le seminorme $\norm{-}_1$ e $\norm{-}_\infty$. Osserviamo che, data una funzione continua $\map{f}{X}{Y}$ fra spazi topologici, i morfismi di complessi indotti $\map{f_\bul}{C_\bul(X)}{C_\bul(Y)}$ e $\map{f^\bul}{C^\bul(Y)}{C^\bul(X)}$ risultano $1$-Lipschitz rispetto alle seminorme appena definite; lo stesso vale dunque per le mappe indotte in (co)omologia. In particolare, segue che le equivalenze omotopiche inducono isomorfismi isometrici in (co)omologia.

Per ogni $n\ge 0$ è ben definita l'applicazione bilineare
\Map{\langle-,-\rangle}{H^n(X)\times H_n(X)}{\RR}{([\varphi],[c])}{\varphi(c),}
detta \defterm{prodotto di Kronecker}. Vale il seguente risultato di dualità.

\begin{proposition}\thlabel{kronecker-product-duality}
Sia $\alpha\in H_n(X)$. Allora
\[
\norm{\alpha}_1=\max\{\langle\beta,\alpha\rangle:\beta\in H^n(X),\norm{\beta}_\infty\le 1\}.
\]
\end{proposition}
\begin{proof}
Una disuguaglianza segue immediatamente osservando che per ogni $c\in C_n(X)$, $\varphi\in C^n(X)$ vale $\norm{a}_1\cdot\norm{\varphi}_\infty\ge\varphi(a)$. Per dimostrare l'altra, fissiamo un ciclo $a$ che rappresenti $\alpha$. Denotiamo con $B_n(X)\subs C_n(X)$ il sottospazio dei bordi. Per Hahn-Banach, esiste un funzionale lineare $\varphi\in C^n(X)$ di norma al più $1$, nullo su $B_n(X)$ e tale che
\[
\varphi(a)=\inf\{\norm{a-b}_1:b\in B_n(X)\}.
\]
Ma allora $\varphi(a)=\norm{[a]}_1$, da cui $\langle[\varphi],\alpha\rangle=\norm{\alpha}_1$.
\end{proof}

\subsection{Volume simpliciale}
D'ora in poi i moduli di (co)catene e di (co)omologia saranno sempre muniti implicitamente delle seminorme $\ell^1$ e $\ell^\infty$ appena definite.

Sia $M$ una $n$-varietà (topologica) chiusa (ossia compatta, connessa e senza bordo) e orientata. È ben noto che $H_n(M,\ZZ)\iso\ZZ$, e l'orientazione permette di distinguere un generatore $[M]_\ZZ\in H_n(M,\ZZ)$, detto \defterm{classe fondamentale} di $M$ a coefficienti interi. L'inclusione di complessi $C_\bul(M,\ZZ)\to C_\bul(M,\RR)$ induce una mappa in omologia $H_\bul(M,\ZZ)\to H_\bul(M,\RR)$; l'immagine di $[M]_\ZZ$ in $H_n(M,\RR)$, indicata con $[M]_\RR$ o semplicemente con $[M]$, è detta \defterm{classe fondamentale} di $M$ (a coefficienti reali).

\begin{definition}
Sia $M$ una $n$-varietà chiusa e orientata. Si definisce \defterm{volume simpliciale} di $M$ il numero reale $\norm{M}=\norm{[M]}_1$.
\end{definition}
Osserviamo che il volume simpliciale non dipende dalla scelta dell'orientazione, dunque è ben definito per qualunque varietà chiusa e orientabile. Se $M$ non è orientabile e $\widetilde M$ è il suo rivestimento doppio orientabile, si definisce $\norm{M}=\lVert\widetilde M\rVert/2$. Infine, se $M$ è compatta e non connessa, si definisce $\norm{M}$ come la somma dei volumi simpliciali delle sue componenti connesse.

Rimandiamo alle sezioni successive il risultato principale di questa trattazione, che permetterà di calcolare il volume simpliciale di varietà Riemanniane piatte o iperboliche. Ci limitiamo per ora a enunciare alcune proprietà del volume simpliciale

Osserviamo innanzitutto che il volume simpliciale dipende solo dal tipo di omotopia della varietà chiusa $M$, come si vede facilmente ricordando che le equivalenze omotopiche inducono isomorfismi in omologia.

\begin{proposition}\thlabel{simplicial-volume-map-degree}
Sia $\map{f}{M}{N}$ una funzione continua fra $n$-varietà orientate, e sia $d$ il grado di $f$. Allora $|d|\cdot\norm{N}\le\norm{M}$.
\end{proposition}
\begin{proof}
Per definizione di grado, vale $H_n(f_\bul)([M])=d\cdot[N]$. Poiché $H_n(f_\bul)$ è $1$-Lipschitz, segue che $|d|\cdot\norm{[N]}_1\le\norm{[M]}_1$, ossia la tesi.
\end{proof}

\begin{corollary}
Se una varietà orientabile $M$ ammette endomorfismi di grado almeno $2$, allora $\norm{M}=0$.
\end{corollary}

\begin{proposition}\thlabel{simplicial-volume-covering}
Sia $\map{f}{M}{N}$ un rivestimento di grado $d$ fra varietà chiuse e orientabili. Allora $\norm{M}=d\cdot\norm{N}$. \todo{Vale anche per varietà non orientabili? Sicuramente $M$ sì, ma $N$?}
\end{proposition}
\begin{proof}
Se $\sum_{i\in I}a_is_i\in C_n(N)$ è un ciclo che rappresenta la classe fondamentale di $N$ in omologia, allora $\sum_{i\in I}\sum_{j=1}^da_i\widetilde{s}_{i,j}\in C_n(M)$ è un ciclo che rappresenta la classe fondamentale di $M$, dove $\widetilde{s}_{i,1},\ldots,\widetilde{s}_{i,d}$ sono i $d$ sollevamenti di $s_i$ \todo{Perché?}. Prendendo l'estremo inferiore delle norme $\ell^1$ al variare dei rappresentanti di $[N]$ otteniamo la disuguaglianza $d\norm{N}\ge\norm{M}$. Per l'altra, è sufficiente osservare che $f$ è una mappa di grado $d$, e applicare la \thref{simplicial-volume-map-degree}.
\end{proof}

Data una $n$-varietà chiusa orientata, esiste un'unica classe in coomologia $[M]^*\in H^n(M)$ tale che $\langle[M]^*,[M]\rangle=1$ \todo{C'è un modo più diretto di vederlo rispetto a costruirla esplicitamente?}; $[M]^*$ è detta \defterm{coclasse fondamentale} di $M$.
\begin{proposition}\thlabel{simplicial-volume-duality}
Vale $\norm{M}=\norm{[M]^*}_\infty^{-1}$ (dove si intende che $\infty^{-1}=0$).
\end{proposition}
\begin{proof}
Dalla \thref{kronecker-product-duality} sappiamo che
\[
\norm{M}=\max\{\langle\beta,[M]\rangle:\beta\in H^n(M),\norm{\beta}_\infty\le 1\}.
\]
Se $\norm{[M]^*}_\infty=\infty$ allora l'unico elemento di $H^n(M)$ di norma finita è $0$, dunque $\norm{M}=0$. Altrimenti è evidente che
\[
\norm{M}=\left\langle\frac{[M]^*}{\norm{[M]^*}_\infty},[M]\right\rangle=\frac{1}{\norm{[M]^*}_\infty}.\qedhere
\]
\end{proof}